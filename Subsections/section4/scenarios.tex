\subsection{Scenario 2}

Tim’s boss contacts him in the evening noticing him his presentation has been delayed to the following day; the location itself has changed, too, forcing Tim to get to the renowned Samsara Beach.\\
Consequently, Tim opens up his Travlendar+ mobile app and reaches for the event ‘presentation to the investors’ in order to modify both time and place.\\
Travlendar+ calculates the distance and the possible travel means, it then informs Tim that the place (while widely known) can’t be reached with public transportation and it isn’t covered by any kind of Car Sharing service (it is out of the influence zone of Travlendar+).\\
The app suggests Tim to go by car. Of course Tim agrees and for the rest of the day, Travlendar+ will consider the car a favourite transport mean.

\subsection{Scenario 3}

David is 17 years old and has recently joined the soccer team of his city. His coach has fixed that training sessions will be held every Monday, Wednesday and Friday at 18:00.\\
Then David opens Travelandar+ and puts his commitments until the end of the season. The application suggests that the fastest way to go to the field training is the subway, so David decides to buy a season pass so that he can safely go to training sessions without having to ask his parents.\\
After buying season pass he registered it on the application.

\subsection{Scenario 4}

Elizabeth loves dedicating the right amount of time to her appointments, from work to family to her hobbies. Recently though she’s having a hard time conciling all of her commitments.\\ 
That’s why her friend Alex recommends her to use the Travelandar + application : Elizabeth follows his advice and downloads the application on her smartphone.\\
It’s only half an hour and Elisabetta is very satisfied, especially because she could set up an "Optimal Lunch" function that allows her to devote the right amount of time to her lunch, denying the opportunity to add appointments around the 30 minutes dedicated to lunch.

\subsection {Scenario 5}
John had an hard day, he’s just put the finishing touches on his project : he had to work even during this weekend, locked at home. Just when he’s done with his assignment he receives an invitation to go out and see a movie with Jane and the rest of his friends.\\
He’s in a rush and he hasn’t previously registered such an appointment in his calendar; to make things worse, he doesn’t own a car, and public transportation is rather slow in the weekend. Because of this he rules out both car and the public transportation as travel means, and when he inserts location and time of the unexpected appointment only car sharing pops up as a viable and fast option.\\ 
John obviously accepts and rents the car through Travlendar+. 

\subsection {Scenario 6}

Luca is a nature-loving person, very passionate about environment and its well-being.\\
He decided to download Travlendar+ because it offers the possibility to minimize the carbon footprints and the usage of its own car.\\
Since he lives in Milan Luca’s usually lucky enough to rent and move with a bike of a shared-system; he’s already chosen to go on foot or by bike as preferred travel means but unfortunately there’s no way to predict whether a bike will be available or not at a given time.\\
When the sun’s up and the new day starts he checks on his phone and verifies that the nearest bike is definitely not worth the trip : he decides to go to work on foot.


\subsection{Scenario 7}

Tom is a bank employee working in Bologna. He decided to return at home in Milan next Friday to celebrate his father's birthday together with his family.\\
Because of this he opens Travlendar+ on his smartphone and creates a "Dad's party" event for Friday night.\\
Tom’s job does not allow him to leave Bologna before 18.00. Fortunately Travlendar+ also allows him to find travel solutions by cross-region trains as well as by car.\\
In fact, it all comes down to Tom's choice. He proceeds to buy train tickets: to him Friday isn’t coming fast enough!
