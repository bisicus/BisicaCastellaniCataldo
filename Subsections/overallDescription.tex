\subsection{Product perspective}

Travlendar+ is a mobile appllication that’s going to be built from scratch and heavily reliant on external APIs. 
Generally, we’ll refer to open and immediately accessible APIs. The system will be able to implement and expand upon iterative additions of external services as its zona d’influenza (glossario) grows and new partnerships arise.
In case an API is not official

In its firts version, Travlendar+ will make extensive use of the following APIs:
\begin{description}

	\item[Google Maps] : to track actual distances and expected times for travels on foot, by car and with public transportation (that shall include trains and inter-city travels too) 
	\href{https://developers.google.com/maps/}{Google Maps}
	\item[Sharing Services] to recover information about availability, parkings and to rent a ride.
	
	\begin{itemize}
		\item \href{https://github.com/car2go/openAPI}{Car2Go}
		\item \href{https://github.com/Autarc/drive-now}{Drive Now}
		\item \href{https://github.com/pierlauro/bikemi-unofficial-api}{BikeMi}
	\end{itemize}
	
\end{description}


			
\subsection{Product functions}
	
 We’ve already detailed the main goals of our mobile application in the first section of the current document; naturally, the aforementioned goals will be functions of Travlendar+.
Here though we submit an higher-level description of major functions to ease the understanding of our program to all interested parties.
		
		\begin{description}
			\item[Appointments Manager] 
			Our mobile application will let any registered user to view, create, modify and delete events, denoted by their type, their date and location.

			\item[Travel Scheduling] 
			When a new appointment is inserted in the application timetable an immediate check is performed whether the location of the event can be reached in the allotted time on foot, by car, or by public transportation (referencing, in lack of additional information, to their default behaviour according to the date and time of the day).
			Day by day, events and allotted times are periodically checked with live informations to ensure they’re feasible : a list of results is prompted to the user whenever he asks for it and he can express preferences and filter them.
			
			\item[Payment Manager for Public Transportation]
			Our mobile application will redirect registered users to payment portals through secure channels and it’ll be able to verify whether transaction succeeded or not by interfacing with external payment services.
			Unfulfilled payment will result in an impossibility to go forward and will prompt the possibility to submit the data and restart the transaction.

			\item[Car and Bike Sharing Integration]
			The mobile app will be able to integrate external info about the availability and location of shared cars and bikes so that it will also allow their reservation.
			Everything past this mark clearly outgrows the scope of our system and it shall be handled by the company that grants the service (same goes for eventual malfunctions or erratic behaviours). 
			Travlendar+ will send send reservation requests to the selected company and will be able to record a rental acceptance notification (i.e., everything’s gone right in the provider’s rental system).


			\item[Free Time spans]
			Our system will grant the possibility to reserve free time spans :
			such breaks won’t be affected by scheduling and by the need to travel and registered users will be able to insert theme directly in the calendar.

	\end{description}

			
	\subsection{User characteristics}
		%cose a caso 2	
			
	\subsection{Assumptions, dependencies and constraints}
		%si spiega da sola
