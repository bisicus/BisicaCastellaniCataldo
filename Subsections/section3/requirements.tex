We now adopt a goal-based approach to determine the requirements associated with each one of the goals we have elaborated in Chapter 1.\\
We'll start numbering and exploring the goals we submitted.

\begin{itemize}

            \item \textit{[G1]} System allows guest user to register with an username ad and a password; to complete the procedure user should confirm by 
               
                  \begin{itemize}
                        \item [R.1.1] System lets registering user choose an username and password
                        \item [R.1.2] Every username corresponds to a single user
                        \item [R.1.3] Duplicate usernames aren’t allowed
                        \item [R.1.4] Registering user can't be already registered
                        \item [R.1.5] An unregistered user is locked out the application and can only see registration page
                        \item [R.1.6] User has to confirm by mail his registration
                  \end{itemize}
             
\item \textit{[G2]} System Login

                  \begin{itemize}
                        \item [R.2.1] User must be already registered to perform correct login
                        \item [R.2.2] User must remember username and password
                        \item [R.2.3] Only a correct combination of username and password will grant access
                        \item [R.2.4] Application will implement a password retrieval mechanism
                  \end{itemize}
                  
\item \textit{[G3]} Registered User can create appointments 

 \begin{itemize}
                        \item [R.3.1] User has to be registered and logged in the system in order to create an
appointment
                        \item [R.3.2] Appointments can be divided into work appointments (or meetings) and personal appointments
                        \item [R.3.3] Appointments require a location and a starting time and an end time
                        \item [R.3.4] Appointments location must be within the boundaries of the operative zone
                        \item [R.3.5] The chosen location can be within the boundaries of the influence zone
                        \item [R.3.6] There cannot be appointments with the same name, location and time
                        \item [R.3.7] System must checks suitability of created new entries based on already existing appointments
                        \item [R.3.8] Appointment start time can't precede the actual system time at the moment of inserting it                                 \item [R.3.9] User can select favourite travel means for each appointment
                        \item [R.3.10] Each appointment must be associated to a level priority
                        
                  \end{itemize}
                  
\item \textit{[G4]} Registered Users can edit meetings

                  \begin{itemize}
                       \item  [R.4.1] A modified meeting must respect all the constraints imposed during the creation of a new meeting, as the requirements in [G3]
                       \item [R.4.2] A meeting can be modified up until its end time
                       \item [R.4.3] If the location of the meeting is modified , the system behaves as if such an event was inserted for the first time, calculating all possibile conflicts with pre-existing event
                       \item [R.4.4] No limit actually exists on the amount of times an event can be modified                     

                 \end{itemize}

\item \textit{[G5]} The application can automatically compute a personalized selection of travel times between appointments to choose from

                  \begin{itemize}
                        \item [R.5.1] The application must refer to Travel Logic for the expected travel time 
                        \item [R.5.2] The application can suggest a combination of various means to reach the desired destination
                        \item [R.5.3] The application must rank the suggestion according to their priority 
                        \item [R.5.4] The registered user can choose to filter out specific travel means
                        \item [R.5.5] Favourite travel means associated to an appointment must always show up
                        
                  \end{itemize}
                  
\item \textit{[G6]} User can choose a preferred solution among the best ones 

                   \begin{itemize}
                        \item [R.6.1] User must be able to choose between ranked solutions : a choice must hide alternatives and give more details
                        \item [R.6.2] The application must arrange a navigable view of feasible solutions
                        \item [R.6.3] 
                  \end{itemize}
                  
\item \textit{[G7]} The application warns the user if locations are unreachable in the allotted time

                   \begin{itemize}
                        \item[R.7.1] The application must realize there's not enough time when the meeting is created
                        \item [R.7.2] The application must use as a reference the temporal distance between the start of                        
                   
                   \end{itemize}

\item \textit{[G8]} Allow users to put constraints on different travel means and limit carbon footprints

\item \textit{[G9]} The application features additional user’s privileged time spans 

\item \textit{[G10]} The application allows to arrange the trips : tickets for public services

\item \textit{[G11]} The application allows the nearest shared vehicle to be found and reserved

                   \begin{itemize}
                        \item [R.11.1] A shared vehicle must necessarily belong to a bike-sharing service or a car-sharing service
                        \item [R.12.2] All services linked to shared vehicles must be automatically disabled if the location of a meeting out of the boundaries of the influence zone
                        \item [R.12.3] All sharing services have their own API and they must be referenced by our mobile application
                        \item [R.12.4] To find or reserve a vehicle it is required by our system the access to the external API of the required service
                        \item [R.12.5] To find or reserve a vehicle it's required that the user logins into the external corresponding service
                        \item [R.12.6] The external service can communicate with our mobile application. In case of reservation Travlendar+ checks if the mobile app corresponding to the desired services is installed on the system. All the following steps take place within such an environment, until control is returned to Travlendar+
                        \item [R.12.7] The location of all the vehicles are shown in the same view, merging data from different APIs
                        \item [R.12.8] When an user decides to rent a car it must be redirected to the corresponding vehicle sharing service, until he's ultimated the rental and gets redirected back to Travlendar+
                        \end{itemize}

\item \textit{[G12]} The application integrates a map system
                   
                  \begin{itemize}
                        \item [R.12.1] The map must be submitted by an external service (in our case we'll select Google APIs)
                        \item [R.12.1] User can navigate through the map indipendently from its current location
                        \item [R.12.3] User can search for a specific location
                        \item [R.12.2] The mobile device can track its current position through geo-localization
                        \item [R.12.3] Position of out the operative zone can't be accepted by the system and won't be displayed
                   \end{itemize}


            \end{itemize}



    
