
\subsection{Functional Requirements}

We now adopt a goal-based approach to determine the requirements associated with each one of the goals we have elaborated in Chapter 1.\\
We'll start numbering and exploring the goals we submitted.

\begin{itemize}

            \item \textit{[G1]} System allows guest user to register with an username ad and a password; to complete the procedure user should confirm by 
               
                  \begin{itemize}
                        \item [R.1.1] System should let registering user choose an username and password
                        \item [R.1.2] Every username corresponds to a single user
                        \item [R.1.3] Duplicate usernames aren’t allowed
                        \item [R.1.4] Registering user can't be already registered
                        \item [R.1.5] An unregistered user is locked out the application and can only see registration page
                        \item [R.1.6] User has to confirm by mail his registration
                  \end{itemize}
             
\item \textit{[G2]} System Login

                  \begin{itemize}
                        \item [R.2.1] User must be already registered to perform correct login
                        \item [R.2.2] User must remember username and password
                        \item [R.2.3] Only a correct combination of username and password will grant access
                        \item [R.2.4] Application will implement a password retrieval mechanism
                  \end{itemize}
                  
\item \textit{[G3]} Registered User can create meetings 

 \begin{itemize}
                        \item [R.3.1] User has to be registered and logged in the system in order to create an
appointment
                        \item [R.3.2] Appointments can be divided into work appointments (or meetings) and personal appointments
                        \item [R.3.3] Appointments require a location and a date
                        \item [R.3.4] Appointments location must be within the boundaries of the operative zone
                        \item [R.3.5] The chosen location can be within the boundaries of the influence zone
                        \item [R.3.5] There cannot be appointments with the same name, location and time
                        \item [R.3.6] Based on already existing appointments, system checks suitability of created new entries
                        \item [R.3.7] Appointment date can't precede the date of insert
                        
                  \end{itemize}
                  
\item \textit{[G4]} Registered Users can edit meetings

                  \begin{itemize}
                       \item  [R.4.1] A modified meeting must respect all the constraint imposed during the creation of a new meeting, as in [G3]
                     


                  \end{itemize}

\item \textit{[G5]} The application can automatically compute a personalized selection of travel times between appointments to choose from

                  \begin{itemize}
                        \item [R.5.1] System verifies the travel mean is feasible for the submitted appointments
                        \item [R.5.2] According to the type of appointment, the system submits the data to corresponding external services
                        \item [R.5.3] Based on meeting type and time of day system ranks all the solutions
                        \item [R.5.4] Time is calculated by imposing a start position submitted by the user as specificied in [G12]
                  \end{itemize}
                  
\item \textit{[G6]} User can choose a preferred solution among the best ones 

                   \begin{itemize}
                        \item [R.6.1] User must be able to choose between ranked solutions
                        \item [R.6.2] The application arranges a navigable view of feasible solutions
                       
                  \end{itemize}
                  
\item \textit{[G7]} The application warns the user if locations are unreachable in the allotted time 

\item \textit{[G8]} Allow users to put constraints on different travel means and limit carbon footprints

\item \textit{[G9]} The application features additional user’s privileged time spans 

\item \textit{[G10]} The application allows to arrange the trips : tickets for public services

\item \textit{[G11]} The application allows the nearest shared vehicle to be found

\item \textit{[G12]} The application can obtain the position of the device and consequently of its user
                   
                  \begin{itemize}
                        \item [R.12.1] User can manually insert a location
                        \item [R.12.2] The mobile device can track its current position through geo-localization
                        \item [R.12.3] Position of out the operative zone won't be accepted by the system
                   \end{itemize}


\item \textit{[G13]} EVENTUAL NAVIGATION


    
