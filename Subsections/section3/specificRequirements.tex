%%% A %%%
\subsection{External Interface Requirements}

	\subsubsection{User Interfaces}				
		The interface of our System is thought to be used via mobile app, because all the functionality make sense only in a \textit{movable} context (meaning, such that the user can do them anywhere they have an internet connection).

We will now list some of the user interfaces thought for \textit{Travlendar+}":

% login %
\begin{figure}[h]
	\includegraphics[width=5cm, height=9.5cm]{mockup/login}
	\centering
	\caption{Login Page.}
	\label{fig:login}
\end{figure}
As mentioned before, a \textit{Guest} or a \textit{non logged User} will encounter page showed in figure \ref{fig:login}, and he will never enter into the app until he completes the Registration or the Login procedures (see sections \ref{register_useCase} and \ref{login_useCase}).
% home page %

\begin{figure}[H]
	\begin{subfigure}{0.5\textwidth}
		\includegraphics[width=5cm, height=9.5cm]{mockup/homepageMonth} 
		\centering
		\caption{Home Page in Monthly view.}
		\label{fig:homePage_Month}
	\end{subfigure}
	\begin{subfigure}{0.5\textwidth}
		\includegraphics[width=5cm, height=9.5cm]{mockup/homepageDaily} 
		\centering
		\caption{Home Page in Daily view.}
		\label{fig:homePage_Day}
	\end{subfigure}
	\caption{Samples of view events.}
\end{figure}
The user can use two different views to manage events in his homepage. Theese are rappresented in figure 2. On the left the application shows monthly view, on the right the daily view. Thus user will modify the scheduling in a simple way by clicking on the events

% create Event %
\begin{figure}[h]
	\includegraphics[width=5cm, height=9cm]{mockup/createAnEvent}
	\centering
	\caption{Create an Appointment page.}
	\label{fig:createEvent}
\end{figure}


% travel Logic %

\begin{figure}[H]
	\caption{Samples of the Travel logic module.}
	
	\begin{subfigure}{0.5\textwidth}
		\includegraphics[width=5cm, height=9.5cm]{mockup/solutions} 
		\centering
		\caption{Solutions found for an Event view.}
		\label{fig:solutions}
	\end{subfigure}
	\begin{subfigure}{0.5\textwidth}
		\includegraphics[width=5cm, height=9.5cm]{mockup/buyATicket} 
		\centering
		\caption{Buy a public transportation ticket view.}
		\label{fig:buyTicket}
	\end{subfigure} \\
	\begin{subfigure}{\linewidth}
		\includegraphics[width=5cm, height=9.5cm]{mockup/reserveCar} 
		\centering
		\caption{Reserve a Sharing service resource view.}
		\label{fig:reserveCar}
	\end{subfigure}

	\label{fig:travelLogic}
\end{figure}

\vfill
	
	\subsubsection{Hardware Interfaces}
		The system allows to interact with all the possible processors that are in the devices of the two companies Android and Apple. Among the most important processors there are: Qualcomm, ARM and Exynos.

	\subsubsection{Software Interfaces}
		\hfill		
		\begin{enumerate}
			\item Android
				\begin{itemize}
					\item[-] Name: Lollipop
					\item[-] Version: 5.0+
					\item[-] Source: \url{https://www.android.com/intl/it_it/versions/lollipop-5-0/}
				\end{itemize}
								
			\item Apple
				\begin{itemize}
					\item[-] Name: iOS
					\item[-] Version: 8.0+
					\item[-] Source: \url{https://www.apple.com}
				\end{itemize}
								
			\item Google Maps APIs
				\begin{itemize}
					\item[-] Name: Google Maps APIs
					\item[-] Source: \url{https://developers.google.com/maps/}
				\end{itemize}
		\end{enumerate}
			
	\subsubsection{Communication Interfaces}
		The system also interface with the applications for the use of sharing and payment services
		\hfill
		\begin{tabular}{| c | c | c |}
			\hline
			Protocol	& Application	& Port \\
			\hline
			\hline
			TCP		& HTTP		& 443 \\
			\hline
			TCP		& HTTP		& 80 \\
			\hline
		\end{tabular}
	
%%% B %%%						
\subsection{Functional Requirements}
	
\subsection{Functional Requirements}

We now adopt a goal-based approach to determine the requirements associated with each one of the goals we have elaborated in Chapter 1.\\
We'll start numbering and exploring the goals we submitted.

\begin{itemize}

            \item \textit{[G1]} System allows guest user to register with an username ad and a password; to complete the procedure user should confirm by 
               
                  \begin{itemize}
                        \item [R.1.1] System should let registering user choose an username and password
                        \item [R.1.2] Every username corresponds to a single user
                        \item [R.1.3] Duplicate usernames aren’t allowed
                        \item [R.1.4] Registering user can't be already registered
                        \item [R.1.5] An unregistered user is locked out the application and can only see registration page
                        \item [R.1.6] User has to confirm by mail his registration
                  \end{itemize}
             
\item \textit{[G2]} System Login

                  \begin{itemize}
                        \item [R.2.1] User must be already registered to perform correct login
                        \item [R.2.2] User must remember username and password
                        \item [R.2.3] Only a correct combination of username and password will grant access
                        \item [R.2.4] Application will implement a password retrieval mechanism
                  \end{itemize}
                  
\item \textit{[G3]} Registered User can create meetings 

 \begin{itemize}
                        \item [R.3.1] User has to be registered and logged in the system in order to create an
appointment
                        \item [R.3.2] Appointments can be divided into work appointments (or meetings) and personal appointments
                        \item [R.3.3] Appointments require a location and a date
                        \item [R.3.4] The chosen location must be found into the operative zone
                        \item [R.3.5] There cannot be appointments with the same name, location and time
                        \item [R.3.6] Based on already existing appointments, system checks suitability of created new entries
                        \item [R.3.7] Appointment date can't precede the date of insert
                  \end{itemize}
                  
\item \textit{[G4]} Registered Users can edit meetings

                  \begin{itemize}
                       
                  \end{itemize}

\item \textit{[G5]} The application can automatically compute a personalized selection of travel times between appointments to choose from

                  \begin{itemize}
                        \item [R.5.1] System verifies the travel mean is feasible for the submitted appointments
                        \item [R.5.2] According to the type of appointment, the system submits the data to corresponding external services
                        \item [R.5.3] Based on meeting type and time of day system ranks all the solutions
                  \end{itemize}
                  
\item \textit{[G6]} User can choose a preferred solution among the best ones 

                   \begin{itemize}
                        \item [R.6.1] User must be able to choose between ranked solutions
                        \item [R.6.2] The application arranges a navigable view of feasible solutions
                       
                  \end{itemize}
                  
\item \textit{[G7]} The application warns the user if locations are unreachable in the allotted time 

\item \textit{[G8]} Allow users to put constraints on different travel means and limit carbon footprints

\item \textit{[G9]} The application features additional user’s privileged time spans 

\item \textit{[G10]} The application allows to arrange the trips : tickets for public services

\item \textit{[G11]} The application allows the nearest shared vehicle to be found

\item \textit{[G12]} The application can obtain a geolocalized position of the device it's running on

\item \textit{[G13]} EVENTUAL NAVIGATION


    

	
	\subsubsection{Use Case Diagrams}
		A global picture of the system interaction with actors is provided here by means of use case diagrams. Following, an analysis of the most interesting use case situations derived from scenarios is presented.

	\includegraphics[width=\textwidth]{uml/useCase}

%%% REGISTER %%%	
	\paragraph{Guest registers to \textit{Travlendar+}} \label{register_useCase}
	
		\begin{tabular}{| l | p{0.8\textwidth} | }
			\hline
			\hline
			Actor	&		Guest. \\
			\hline
			Input Condition		&		NULL. \\
			\hline
			Event Flow		&		\begin{enumerate}
												\item Guest clicks on "Sign Up}" button.
												\item Guest fills in at least all mandatory fields.
												\item Guest reads and accepts privacy policies and agreements from the company.
												\item Guest clicks on "Confirm" button.
												\item System sends Guest a confirmation link to the provided e-mail.
												\item Guest clicks on the confirmation link.
												\item	 System saves the data in the DB.
											\end{enumerate} \\
			\hline
			Output Condition		&		Guest succesfully ends registration process and become a User. From now on he/she can log in to the application using his/her credential and start using \textit{Tralvendar+}. \\
			\hline		
			Exception		&		\begin{itemize}
											\item[-] Guest is already a User.
											\item[-] One or more mandatory fields are not valid.
											\item[-] Choosen username is already in use.
											\item[-] Email choosen is already associated to another user.
										\end{itemize}
										All exception are handle alerting the visitor of the problem and application goes back to point 2 of Event Flow \\
			\hline
			\hline
		\end{tabular}

%%% LOGIN %%%	
	\paragraph{Guest logins into \textit{Travlendar+}} \label{login_useCase}
	
		\begin{tabular}{| l | p{0.8\textwidth} | }
			\hline
			\hline
			Actor	&		Guest, User. \\
			\hline
			Input Condition		&		Guest is registered to  \textit{Travlendar+}. \\
			\hline
			Event Flow		&		\begin{enumerate}
												\item Guest fill login mandatory fields.
												\item Guest clicks on "Log In" button.
												\item System verifies login credentials.
											\end{enumerate} \\
			\hline
			Output Condition		&		Guest is promoted to User and is shown is Calendar home page. \\
			\hline		
			Exception		&		Login credentials are incorrect and Guest is shows again Login page\\
			\hline
			\hline
		\end{tabular}

%%% CREATE A NEW APPOINTMENT %%%	
	\paragraph{Create a New Appointment} \label{createEvent_useCase}
	
		\begin{tabular}{| l | p{0.8\textwidth} | }
			\hline
			\hline
			Actor	&		User. \\
			\hline
			Input Condition		&		User is already logged in into \textit{Travlendar+}. \\
			\hline
			Event Flow		&		\begin{enumerate}
												\item User click on "Create Appointment".
												\item User name, sets day, time and position of the Appointment.
												\item System checks if the new appointment overlaps with already existing appointments or break period.
												\item	 System calculates, ranks and shows multiple solutions depending on user travelling preferences.
												\item User selects one solution as preferend one.
											\end{enumerate} \\
			\hline
			Output Condition		&		\textit{Tralvendar+} shows calendar main page with the new appointment. \\
			\hline		
			Exception		&		\begin{itemize}
											\item[-] Created appointment overlaps with already existing appointments.
											\item[-] There are no feasible solution.
											\item[-] Inserted location isn't in the \textit{Operative Zone}.
										\end{itemize} \\
			\hline
			\hline
		\end{tabular}


%%% MODIFY APPOINTMENT %%%

	\paragraph{Modify appointment}
	
		\begin{tabular}{| l | p{0.8\textwidth} | }
			\hline
			\hline
			Actor	&		User. \\
			\hline
			Input Condition		&		\begin{itemize}
													\item[-] User is already logged in into \textit{Travlendar+}.
													\item[-] Appointment already exists.
												\end{itemize} \\
			\hline
			Event Flow		&		\begin{enumerate}
												\item User click on "Appointment".
												\item User starts modifying process.
												\item System checks if the new appointment overlap with already existing appointments or break period.
												\item	 System calculates, ranks and shows multiple solution depending on user travelling preferences.
												\item User selects one solution as preferend one.
											\end{enumerate} \\
			\hline
			Output Condition		&		\textit{Tralvendar+} shows calendar main page, with the updated appointment. \\
			\hline		
			Exception		&		\begin{itemize}
											\item[-] Modified appointment overlaps with already existing appointments.
											\item[-] There are no more feasible solutions.
											\item[-] Modified location is no more in the \textit{Operative Zone}.
										\end{itemize} \\
			\hline
			\hline
		\end{tabular}
		
		
%%% INSERT PAYMENT METHOD%%%

	\paragraph{Insert Payment Method}
	
		\begin{tabular}{| l | p{0.8\textwidth} | }
			\hline
			\hline
			Actor	&		User. \\
			\hline
			Input Condition		&		\begin{itemize}
													\item[-] User is already logged in into \textit{Travlendar+}.
													\item[-] Credit Card isn't already inserted on the system.
												\end{itemize} \\
			\hline
			Event Flow		&		\begin{enumerate}
												\item User click on "Preferences/Payment Methods".
												\item User sets all the credit cards info.
												\item System checks and validate provided informations.
											\end{enumerate} \\
			\hline
			Output Condition		&		\textit{Tralvendar+} returns to \textit{Payment Methods} page showing added card as a valid payment method. \\
			\hline		
			Exception		&		Credit card given informations are invalid. \\
			\hline
			\hline
		\end{tabular}

%%% BUY TICKET %%%

	\paragraph{Buy Public Transportation Ticket} \label{buyTicket_useCase}
	
		\begin{tabular}{| l | p{0.8\textwidth} | }
			\hline
			\hline
			Actor	&		User. \\
			\hline
			Input Condition		&		\begin{itemize}
													\item[-] User is already in "Solutions" page.
													\item[-] A payment method is already available.
												\end{itemize} \\
			\hline
			Event Flow		&		\begin{enumerate}
												\item User clicks on desired solution.
												\item User clicks on "Buy Ticket" button.
												\item System shows available public trasportation tickets.
												\item User selects a ticket.
												\item	 System starts purchase transaction.
											\end{enumerate} \\
			\hline
			Output Condition		&		Based on public transportation service, User receives a valid ticket. \\
			\hline		
			Exception		&		\textit{External Transaction Service} doesn't worked as expected. \\
			\hline
			\hline
		\end{tabular}

%%%  RESERVE SHARING SERVICE %%%

	\paragraph{Reserve a \textit{Sharing Service} resource} \label{sharing_useCase}
	
		\begin{tabular}{| l | p{0.8\textwidth} | }
			\hline
			\hline
			Actor	&		User. \\
			\hline
			Input Condition		&		\begin{itemize}
													\item[-] User is already in "Solutions" page.
													\item[-] A "Sharing mean" is already available.
													\item[-] A "payment method" is already available.
													\item[-] User is in the \textit{influence zone}
												\end{itemize} \\
			\hline
			Event Flow		&		\begin{enumerate}
												\item System connects to available "Sharing Service" resources.
												\item System ranks per distance all the freasible recources and shows them on a map centered on User position.
												\item User chooses one on the possible solutions.												
												\item User clicks on "Reserve it" button.
												\item	 System reconnects to selected \textit{Sharing Service}, and starts the reserving procedure.
											\end{enumerate} \\
			\hline
			Output Condition		&		User redirected to the "Reservation Service" app. \\
			\hline		
			Exception		&		Reservation Service" doesn't worked as expected. \\
			\hline
			\hline
		\end{tabular}
		
		
%%%  SET TRIP PREFERENCES%%%

	\paragraph{Set Trip Preferences}
	
		\begin{tabular}{| l | p{0.8\textwidth} | }
			\hline
			\hline
			Actor	&		User. \\
			\hline
			Input Condition		&		User is already logged in into \textit{Travlendar+}. \\
			\hline
			Event Flow		&		\begin{enumerate}
												\item User click on "Preferences/Trip".
												\item System shows all the possible preferences.
												\item User pins prefered options.
											\end{enumerate} \\
			\hline
			Output Condition		&		\begin{itemize}
													\item[-] User returns to Caledar page.
													\item[-] System recalculate all future trip solution according to User preferences.
													\item[-] User is informed of particular problem
												\end{itemize} \\
			\hline
			Exception		&		User unpins all the possible preferences. \\
			
			\hline
			\hline
		\end{tabular}
		
		
%%%  SET BREAK %%%

	\paragraph{Set Break Period}
	
		\begin{tabular}{| l | p{0.8\textwidth} | }
			\hline
			\hline
			Actor	&		User. \\
			\hline
			Input Condition		&		User is already logged in into \textit{Travlendar+}. \\
			\hline
			Event Flow		&		\begin{enumerate}
												\item User click on "Preferences/Breaks".
												\item User select an interval and a minimum lenght for his break.
											\end{enumerate} \\
			\hline
			Output Condition		&		System adds those hours as a special meeting every day in the calendar. \\
			\hline
			\hline
		\end{tabular}


	\subsubsection{Sequence Diagrams}
		\paragraph{Create Event}
	\includegraphics[width=\textwidth]{uml/sequenceDiagrams/createEvent}
	\vfill
	
\paragraph{Subsquence: Check for Overlapping event}
	\includegraphics[width=\textwidth]{uml/sequenceDiagrams/checkOverlap}
	\vfill
		
\paragraph{Subsquence: Schedule a Trip}	
	\includegraphics[width=\textwidth]{uml/sequenceDiagrams/scheduleTrip}
	\vfill		
			
%%% C %%%
\subsection{Performance Requirements}
This section is useful to see how the application manages statical and dynamical non-functional requirements in terms of quantity and quality. 
	In particular:
	\begin{description}
		\item[Static Non-Functional Requirements]	\vfill
		\hfill
		\begin{enumerate}
			\item Application it was developed to handle until 50.000 users simultaneously.
			\item Mobile application must run on two most importante operative system: iOS and Android.
		\end{enumerate}
		
		\item[Dynamic Non-Functional Requirements] \vfill
		\hfill
		\begin{enumerate}
			\item 98\% of the requestes shall be processed in less than 2 second.
			\item The system should be able 99.8\% of the time in one year.
		\end{enumerate}			

	\end{description}
		
		
%%% D %%%	
\subsection{Design Constraints}
		\begin{description}
			\item[Standards compliance]
			The system must require User’s location to work, so the application asks for it according to privacy laws the first time the user runs the application.
			The payment system is guaranteed by the corresponding external application, so \textit{Travlendar+} doesn’t ask users for their credit card number.

			\item[Hardware limitations]
			Mobile app:
			\begin{itemize}
			\item Space for app package
			\item 3G connection or better
			\item GPS
		\end{itemize}

			\item[Any other constraint]
			No further constraints are imposed
\end{description}
		
%%% E %%%		
\subsection{Software System Attributes}
	Sofware System Attributes (also called non-functional requirements) describe how the system works and allow us to judge system's doing because it receives an high quantity of requests and its performance must be tested under such circumstances.

\subsubsection{Reliability}
The system must be active 24/7 and servers are allowed to be out of service only for system updates and the likes.
 
\subsubsection{Availability}

\subsubsection{Security}

\subsubsection{Maintainability}

\subsubsection{Portability}

