\begin{description}
				\item[User] 
				\begin{itemize}
					\item First name;
					\item Last name; 
					\item Email;
					\item Username;
					\item Password;
					\item Payment information; this in particular includes:
						\begin{itemize}
							\item Credit card owner;
							\item Credit card number;
							\item Credit card expiration date;
							\item CVV number.
						\end{itemize}
				\end{itemize}
				
				\item[Guest] We name 'guests' all the people who are using the interface of the system without being registered or logged in. Guests can't access any functionality of \textit{Travlendar+} except for the registration process and the log in. 
				\item[Operative Zone] We name Operative Zone the area within we can place the location of an event. For the time being the Operative Zone coincide with all the cities and places within italian peninsula. Naturally, such an area may be expanded in the future.
				\item[Influence Zone] We name Influence Zone the area within whose borders the mobile application can not only give travel time by car and on foot (the minimum standard given to us by Google APIs) but also where Travlendar+ can rely on car and bike sharing services. For starting, the Influence Zone will coincide with the city of Milan.
				\item[Mobile Application] By mobile application we refer to a program conceived for Android and iOS operative systems, based on touch interfaces and able to run on portable devices.
				\item[Appointment] An appointment is an event well delimited both in time and space, requiring the presence and the direct investment, in our case, of the user who creates it. Appointments fall in two categories : meetings (work priorities) and personal appointments (the broader set encapsulating all other kinds of appointments, mainly regarding personal and family life)
				\item[Warning] A warning is a notification given by Travlendar+ mobile application to the operating System it is hosted by. It behaves as a standard system notification.
\end{description}
