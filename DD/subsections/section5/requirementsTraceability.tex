The Design Document at hand was thought and developed in the scope of fulfilling optimally the requirements specified in its corresponding RASD. While the RASD gives a comprehensive list of the various Goals and requirements, here we'll only display how the mapping between the two documents was done, showing the bare minimum information concerning the latter document.

\begin{description}
	\item \textit{[G1]}System provides an authentication system.
		\begin{itemize}
			\item [R.1.1] System provides sign-up and an authentication mechanism.
			\textbf{This requirement is mapped into Access Manager component}.
			
			\item [R.1.2] System requires a unique username and a password for every user.			
			\textbf{This requirement is mapped into Travlendar Server component}.
			
			\item [R.1.3] An unregistered user is locked out the application and can only see the registration page.			
			\textbf{This requirement is mapped into Mobile Application and Access Manager components}.

			\item [R.1.4] User has to confirm by mail his registration.			
			\textbf{This requirement is mapped into Travlendar Server component}.
			
			\item [R.1.5] Only a correct combination of username and password will grant access.			
			\textbf{This requirement is mapped into an Access Manager and Travlendar Server component}.

		     \item [R.1.6] Application will implement a password retrieval mechanism.
		     \textbf{This requirement is mapped into Travlendar Server component}.

			\item [R.1.7] Each modification made to a user account must be saved into Travlendar+ Server to be made effective.
			\textbf{This requirement is mapped into Travlendar Server and Access Manager component}.

			\item [R.1.8] New user registration is successful only after data is stored on Travlendar+ Server and a confirmation is received by the system.
			\textbf{This requirement is mapped into Travlendar Server component}.
		\end{itemize}


	\vskip0.75cm
	\item \textit{[G2]} The application integrates a time-slot based system for appointments.
		\begin{itemize}
			\item [R.2.1] The calendar integrates a calendar and a timetable.
			
			\item [R.2.2] Calendar must give to the user granularity regarding both months and days.
			\textbf{These requirements are mapped into the Calendar Manager components}.

			\item [R.2.3] Calendar and Timetable can be modified only by the user inserting events. No one else is allowed to either see or modify the information they contain.			
			\textbf{This requirement is mapped into Calendar Manager and Application Aggregator components}.
			
			\item [R.2.4] Calendar and Timetable for each user are remotely copied on Travlendar+ Server every time a user creates/modifies/deletes an event.
			
			\textbf{This requirement is mapped into Calendar Manager and Travlendar Server components}.
		\end{itemize}

	\vskip0.75cm
	\item \textit{[G3]} Registered User can create appointments.
		\begin{itemize}
			\item [R.3.1] User has to be registered and logged in the system in order to create an appointment.
			\textbf{This requirement is mapped into Calendar Manager and Application Aggregator components}.

			\item [R.3.2] Appointments can be divided into work appointments (or meetings) and personal appointments.

			\item [R.3.3] Appointments require a location, a starting time and an end time.
			\textbf{These requirements are mapped into the Calendar Manager component}.

			\item [R.3.4] Appointments location must be within the boundaries of the operative zone.
			\textbf{This requirement is mapped into Calendar Manager and Travel Logic components}.

			\item [R.3.5] There cannot be appointments with the same name, location and time.
			\textbf{This requirement is mapped into Calendar Manager component}.

			\item [R.3.6] System must check suitability of new entries based on already existing appointments.
			\textbf{This requirement is mapped into Calendar Manager and Travel Logic components}.

			\item [R.3.7] Appointment start time can't precede the actual system time at the moment of inserting it.

			\item [R.3.8] User can select favourite travel means and priority for each appointment.

			\item [R.3.9] Each appointment must be associated to a level priority	.
			\textbf{These requirements are mapped into Calendar Manager component}.

			\item [R.3.10] The creation of an appointment must be remotely saved on Travendlar+ server in order to be successful and complete.
			\textbf{This requirement is mapped into Calendar Manager and Travlendar Server components}.
		\end{itemize}


	\vskip0.75cm
	\item \textit{[G4]} Registered Users can edit appointments.
		 \begin{itemize}
			\item  [R.4.1] A modified meeting must respect all the constraints imposed during the creation of a new meeting, as the requirements in \textit{[G3]}.

		 	\item [R.4.2] A meeting can be modified up until its end time.

		 	\item [R.4.3] If the meeting is modified, the system behaves as if such an event was inserted for the first time, calculating all possibile conflicts with pre-existing events.

		 	\item [R.4.4] No limit actually exists on the amount of times an event can be modified within the aforementioned constraints.
		 	\textbf{These requirements are mapped into Calendar Manager and Application Aggregator components}.

		 	\item [R.4.5] A modification must be correctly saved on the remote \textit{Travlendar+} server in order to be succesful and completed.
			\textbf{This requirement is mapped into Calendar Manager and Travlendar Server components}.
			
		 	\item [R.4.6] Deleting an appointments must belong to the set of modifications.
		 	\textbf{This requirement is mapped into Calendar Manager component}.
		 \end{itemize}


	\vskip0.75cm
	\item \textit{[G5]} The application can automatically compute a personalized selection of travel times between appointments to choose from.
		\begin{itemize}
			\item [R.5.1] The application must refer to Travel Logic for the expected travel time.

			\item [R.5.2] The application must be able to suggest a combination of various means to reach the desired destination.

			\item [R.5.3] In case the trip expects more than one travel mean, the journey must be divided into sub-problems whose expected travel time has to be calculated. Same goes with public means stop and shared vehicles.
			\textbf{These requirements are mapped into Travel Logic component}.

			\item [R.5.4] Starting location for travel can be inserted manually, retrieved by the previous event or calculated through geo-localization.
			\textbf{This requirement is mapped into Travel Logic and Localization Manager components}.

			\item [R.5.5] The application must rank the suggestions according to their priority, presence of preferred travel means and time required.
			\textbf{These requirements are mapped into Travel Logic component}.

			\item [R.5.6] The registered user must be able to choose to filter out specific travel means.			

			\item [R.5.7] Favourite travel means associated to an appointment must always show up.
			\textbf{These requirements are mapped into Travel Logic and Preference Manager components}.

			\item [R.5.8] In case two or more appointments overlap, an appointment with higher priority is considered automatically chosen and all the remaining ones are arranged according to their priority. Warnings must follow as expected.
			\textbf{This requirement is mapped into Travel Logic and Notification Manager components}.

			\item [R.5.9] The route can include intermediate destinations before the final, target one.

			\item [R.5.10] When a shared vehicle is suggested the parking zone nearest to the destination must be always inserted among the intermediate destinations.
			
			\textbf{These requirements are mapped into Travel Logic component}.


			\item [R.5.11] The sytem must grant to know daily scheduled times for public transportation through its APIs.
			\textbf{This requirement is mapped into API Manager component}.

			\item [R.5.12] When the starting time of a trip associated to an event is only one hour away the system must notify the user with an updated list of travel time so he can choose.
			\textbf{This requirement is mapped into Travel Logic and Notification Manager components}.

			\item [R.5.13] According to real world data, each travel must have associated to itself the carbon footprints.
			\textbf{This requirement is mapped into Travel Logic and Preference Manager components}.

			\item [R.5.14] Travels that do not satisfy all User's contraints must be excluded.			
			\textbf{This requirement is mapped into Travel Logic and Preference Manager components}.
		\end{itemize}
		


	\vskip0.75cm
	\item \textit{[G6]} User can choose a solution among the scheduled ones. 
		\begin{itemize}
			\item [R.6.1] Selecting a solution that is not a personal vehicle must show both intermediate and final destinations.
			\textbf{This requirement is mapped into Calendar Manager and Travel Logic components}.

			\item [R.6.2] The application must arrange a navigable interface of feasible solutions.
			\textbf{This requirement is mapped into Mobile Application, Application Aggregator and Calendar Manager components}.

			\item [R.6.3] Choosing a solution that includes a public transportation mean must show the user the possibility to buy a ticket. In case of ticket purchase \textit{Travlendar+} checks if the mobile app corresponding to the desired services is installed on the system. All the following steps take place within such an environment, until control is returned to \textit{Travlendar+}.
			\textbf{This requirement is mapped into Travel Logic components}.

			\item [R.6.4] Choosing a solution that includes a shared vehicle must show the user the possibility to locate and rent such a vehicle.
			\textbf{This requirement is mapped into Travel Logic and API Manager components}.

			\item [R.6.5] Choosing a solution must not be definitive.
			\textbf{This requirement is mapped into Calendar Manager and Travel Logic components}.

			\item [R.6.6] System must recognize by itself through geolocalization that a user reached destination; also, User must always be able to stop the trip.
			\textbf{This requirement is mapped into Localization Manager and Travel Logic components}.
		\end{itemize}



	\vskip0.75cm
	\item \textit{[G7]} The application warns the user if locations are unreachable in the allotted time.
		\begin{itemize}
			\item[R.7.1] The application must realize if the alloted time is sufficient from either the last event, current location or manually inserted location.
			\textbf{This requirement is mapped into Travel Logic and Localization Manager components}.

			\item[R.7.2] The application must use as a reference the time to cover distance between the starting place and the destination one, using the futured scheduled time for public transportation if necessary.
			\textbf{This requirement is mapped into Travel Logic and API Manager components}.

			\item [R.7.3] Warning must arrive also while on the road if the travel mean is no longer suitable, or the best solution: in that case the system is going to prompt a new eventual choice of travel means.
			\textbf{This requirement is mapped into Notification Manager, Travel Logic, and Localization Manager components}.

			\item [R.7.4] When user reaches destination warnings must stop automatically.
			\textbf{This requirement is mapped into Notification Manager and Localization Manager components}.

			\item [R.7.5] Warnings can be disabled on the road by the user.
			\textbf{This requirement is mapped into Application Aggregator component}.
		\end{itemize}


	\vskip0.75cm
	\item \textit{[G8]} Allow users to put constraints on different travel means and limit carbon footprints.
		\begin{itemize}
			\item[R.8.1] User must be able to rule out vehicles from search result returned by the system scheduler.

			\item[R.8.2] When the option of limiting carbon footprints gets enabled the associated CO2 consumed by each travel must be taken into account in travels scheduling.

			\item[R.8.3] User must be able to put a constraint on the number of travel means adopted for a single travel.

			\item[R.8.4] User must allow at least a single travel mean.

			\item[R.9.5] User cannot remove "walking" from travel mean preferences.		
			\textbf{All these requirements are mapped into Travel Logic and Preference Manager components}.
		\end{itemize}


	\vskip0.75cm
	\item \textit{[G9]} The application features additional User’s breaks.
		\begin{itemize}
			\item [R.9.1] Each Break is characterized by a duration, the time of the day they start in and by the time frame within are allowed.

			\item[R.9.2] Breaks can be periodic.

			\item[R.9.3] System reserves a minimum quantity of time which is not shorter than the break duration.

			\item[R.9.4] Breaks must be completely encapsulated within the time frames the break is allowed in.
			\textbf{These requirements are mapped into Calendar Manager component}.
			
			\item[R.9.5] Within the time frame of a break the scheduler must always grant a free time span whose duration must be at least equal to the corresponding break duration.
			\textbf{This requirement is mapped into Calendar Manager and Travel Logic components}.
		\end{itemize}


	\vskip0.75cm
	\item \textit{[G10]} The application allows to buy tickets for public services.
		\begin{itemize}
			\item[R.10.1] Buying a ticket must reroute the user to the corresponding mobile application, after system has succesfully checked it is installed.

			\textbf{This requirement is mapped into Payment Manager component.}
			
			\item[R.10.2] Purchase confirmation and additional data about the payment is retrieved through API requests.
			
			\textbf{This requirement is mapped into API Manager and Payment Manager components}.
		\end{itemize}


	\vskip0.75cm
	\item \textit{[G11]} The application allows the nearest shared vehicle to be found and reserved.
		\begin{itemize}
			\item [R.11.1] A shared vehicle must necessarily belong to a bike-sharing service or a car-sharing service.

			\item [R.11.2] All services linked to shared vehicles must be automatically disabled if the location of an appointment is out of the boundaries of the influence zone.

			\textbf{These requirements are mapped into API Manager and Travel Logic components}.


			\item [R.11.3] All sharing services have their own API which is used by the system to locate the vehicles and retrieve confirmation of reservation.

			\textbf{This requirement is mapped into API Manager component}.

			\item [R.11.4] The external service can communicate with our mobile application. In case of reservation \textit{Travlendar+} checks if the mobile app corresponding to the desired services is installed on the system. All the following steps take place within such an environment, until control is returned to \textit{Travlendar+}.			
			\textbf{This requirement is mapped into Travel Logic component}.

			\item [R.11.5] The location of all the vehicles must be shown in the same interface, merging data from different APIs.

			\item[R.11.6] Only shared vehicles that are free and available must be displayed and possibly reserved.
			\textbf{These requirements are mapped into API Manager component}.
		\end{itemize}


	\vskip0.75cm
	\item \textit{[G12]} The application allows the user to oversee his position in real-time as well as the route of his travel.
		\begin{itemize}
			\item [R.12.1] Application integrates a map system submitted by GMAPS API.

			\item [R.12.2] User must be able search for a specific location.
			\textbf{These requirements are mapped into API Manager and Travel Logic components}.

			\item [R.12.3] The mobile device must be able to track its current position through geo-localization.
			\textbf{These requirements are mapped into API Manager and Localization Manager components}.

			\item [R.12.4] Positions out of the operative zone can't be accepted by the system and won't be displayed.			
			\textbf{This requirement is mapped into Travel Logic component}.

		\end{itemize}


	\vskip0.75cm
	\item \textit{[G13]} The User can submit additional preferences.
		\begin{itemize}
			\item[R.13.1] User must be able to forbid travel means within time spans, also periodical ones.

			\item[R.13.2] User must be able to put a constraint on the maximum amount of space and time he can give to each travel mean.

			\item[R.13.3] User must be able to link one or more season passes to his account.
			\textbf{These requiremente are mapped into Preference Manager component}.
			
			\item[R.13.4] User must be able to link one or more credit cards to his account.
			\textbf{This requirement is mapped into Payment Manager component}.

			\item[R.13.5] Each modification apported by the User to its additional preferences is only made effective when synced on \textit{Travlendar+} Server.
			\textbf{This requirement is mapped into Preference Manager and Travlendar Server components}.
		\end{itemize}
\end{description}