For our system architecture we adopt a 3-tier client/server architecture : we'll have a Mobile Application Clients (the presentation layer) and Travlendar Server (which accounts with its separate functions both for Logic layer and Data Layer).


\paragraph{Mobile Application Clients:} This layer is represented by the \textit{Travlendar+} Application, a thick client and an interactive and dinamic GUI.
The application will be written in Java for increased portability and will be able both to geo-localize its user and to communicate with the core of the travel Logic: this is the foundation of information synching and requests forwarding to APIs.

\paragraph{\textit{Travlendar+} Server:} The business Server runs the business logic. Its main duty is to forward requests of the users to the Data layer and external APIs and to send them back bound by the necessary users' constraints. These operations are executed under a microservice architectural pattern.

\paragraph{The Data Layer:} This layer comprehends both the DBMS -which stores and manages users' preferences and timetables- and the external service agents (objects which call external services). In the former case it is required the capability of providing data through SQL queries using ODBC protocol, while in the latter it is required a good implementation of the ad-hoc functions of the external APIs.

The mobile application will be developed in Java because of the cross-compatibility and portability of the language.
The Travlendar Server will be developed in JEE and will link single users with the database.
The DBMS will be managed through MySQL relational language.

\subsubsection*{Patterns} 

Pattern are mostly required to smooth the construction of our system. In particular, we'll be using a MVC (Model - View - Control) Pattern for our mobile application, Client/Server and Single Instance component.

MVC is used to define our Java mobile application: the view will be the dynamic GUI, the Travel Logic the control and the model will be the data received from the APIs and the locally stored timetables and trips.

Single Instance component will be mostly useful when treating the components of the mobile application: components like Access Manager or Travel Logic will be the ones mainly touched by this pattern.

Client/Server pattern is the one we adopted to shape the interactions between our mobile application (the client) and the Travlendar+ Server (i.e, the server). Reasons for adoption have mainly been the simplicity and widespread use of the model.
