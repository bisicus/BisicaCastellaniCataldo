In this section we will describe the most interesting algorithms, from a computational point of view; we have decided not to include algorithms for the more mechanical and common part of the system (such as login operations or interactions with the DBMS). In addition, some sequence of actions have already been described in a more general way with some Sequence Diagrams in previous parts of this document, and in the \textit{Use Cases} section of the RASD.

% RANK SOLUTIONS %
\subsection{Ranking}

This algorithm shows how System ranks all the feasible solution based provided by the external sources, like Public Transportation, via Car, Bike or Foot, depending on:
	\begin{itemize}
		\item[•] time needed for completing the trip.
		\item[•] number of subtrips (only for Public Tranportation).
		\item[•] if the Car have to be used in other trips on that day.
		\item[•] money needed for the Tranportation Mean, calculated as:
			\begin{itemize}
				\item[-] public ticket cost for Public Services.
				\item[-] average fuel cost for Car.
				\item[-] tariff \euro /time for Sharing services.
			\end{itemize}
		\item[•] User preferences.
		\item[•] Weather forecast for the Appointment day (if and only if the Appointment is no longer than 15 days).
	\end{itemize}
	
	Since solutions have been already calculated individually, is supposed that every element in the input list has been already divided in subtrips either by the External API Manager or by Scheduler (see figure \ref{travelLogicDetail} and figure \ref{APIManagerDetail}) in a previous Step.
	Is also assumed that 'Public Service Manager' (see RASD Class Diagram) provides a list of different tranportation means, and all the consistent possible combinations of them.
	Every element in the input list is filtered by the 'Excluded Vehicles List' (see figure \ref{preferenceManagerDetail})
	For remaining elements, every inner SubTrip time is compared with the average reference time that the User have to spend by going with owned Car or bike, or, if is a short distance, by going with foot, in order to categorize solutions in \textit{suitables}, \textit{valid alternatives}, and \textit{unconvenient}.
	
	Then for every category solutions:
	\begin{itemize}
		\item[-] items are orderd by \textit{time needed} and \textit{number of SubTrip}
		\item[-] if User has a Season Pass of a public transportation company, relative solutions are put on the highest rank.
		\item[-] solution cost are calculated, as aforementioned.
	\end{itemize}		
		
	If a solutions from lower category are advantageous with respect to 'higher' solutions, those are put again in the upper category.
	A solution is advantageous if \textit{time needed}s difference between the two category, defined as 
	\\ \quad $\Delta =  timeNeeded(bestUpperSolution) - timeNeeded(lowerSolution)$, \\
	is less than 15 minutes, and either is a cheaper solution or User has a Season Pass that doesn't belong to any other company of Upper-category solutions.
		
	If the Appointment is scheduled in less than 15 days, and weather forecast is predicted al \textit{non consistent for outdoor tripping}, solutions that expect a $total\_Outdoor\_Time$ defined as the sum of the time spent by walking and biking, greater than 3 minutes are downgraded in the respective category.

	If in the other Appointments of that day a proprietary Car solution is already scheduled, 'Only by Using Car' solution will be encouraged, but if and only if:

	In the end all the category \textit{suitables} and \textit{valid alternatives} are merged into an unique list made so that User can choose one.
	The algorithm return also the $bestSolution$ got by 'Poping' of the first element of the list, and the solution obtained by 'Only Walking' and 'Only by Using Car'.
	
	\vfill
	\begin{algorithm}[H]
\caption{Rank Solution}
	\KwData{Trip, List of Calculated Solutions, Preferences, Calendar}
	\KwResult{List of Ranked Solutions, Best Calculated Solution, Only Car Time, Only Walking Time}
		
	\bigskip
	\tcp{Calculate 'Only Car' and 'Only Walking' Solutions}
	$ carSolution \leftarrow \textbf{Scheduler}.carTime$\;
	$ bikeSolution \leftarrow \textbf{Scheduler}.walkingTime$\;
		
	\bigskip
	\tcp{Filtering Solutions based on $Preferences$}	
	\ForAll{Trip in Input\_List}{
		\If{Trip.AssignedTransportationMean() is in Excluded\_Vehicles\_List}{
			$ InputList.remove(Trip) $\;
		}
	}
	$ FilteredList \leftarrow InputList $\;

	\bigskip
	\tcp{Assign a category}	
	\ForAll{Trip in Filtered\_List}{
		$ \delta_{Advantage} \leftarrow 0$\;
		\ForEach{SubTrip in Trip}{
			\Switch{SubTrip}{
				\Case{Long\_Distance}{
					$ AVG\_Ref \leftarrow 
						\quad \arg\min
							\Big( \textbf{Scheduler}.AVG\_Time(Car), \textbf{Scheduler}.AVG\_Time(Bike) \Big)$\;
				}
				\Case{Short\_Distance}{
					$ AVG\_Ref = \textbf{Scheduler}.AVG\_Time(Foot)$\;
				}
			}
			$ comparison \leftarrow AVG\_Ref - SubTrip.Time$\;
			$ \delta_{Advantage} \leftarrow \delta_{Advantage} + comparison$\;
		}
			
		\bigskip
		\If{$ \delta_{Advantage} \gg 0 $ }{ 
			$ Suitable\_List.add(Trip)$\;
		}
		\ElseIf{$ \delta_{Advantage} \geq 0 $ }{
			$ Valid\_Alternatives\_List.add(Trip)$\;
		}
		\ElseIf{$ \delta_{Advantage} < 0$ }{
			$ Unconvenient\_List.add(Trip)$\;
		}
		\Else{
			$ Filtered\_List.remove(Trip)$\;
		}
	}
\end{algorithm}
\vfill
\begin{algorithm}[H]
%reprise of the algorithm %
\setcounter{AlgoLine}{24}
	\bigskip
	\tcp{Category Ranking}
	\ForEach{Category\_List}{
		$ Category\_List.sortBy(TimeNeeded, SubTripsNumber, ascending)$\;
		\ForAll{Trip in $Category\_List$}{
			$ cost \leftarrow calculateCost(Trip)$\;
			\If{User has Season Pass for that Trip}{
				$ cost \leftarrow SeasonPass.Promotion$\;
			}
		}
	}
	$ Category\_List.partialOrdering(cost) $ \;
		
	\bigskip
	\tcp{Searching for Advantageous Lower Solutions}
	$ bestSolution \leftarrow Suitable\_List.takeFirst()$\;
	\ForEach{lowerTrip in Lower\_Categories\_List}{
		$\Delta \leftarrow  timeNeeded(bestSolution) - timeNeeded(LowerTrip)$\;
		\If{$\Delta \leq 15$ minutes}{
			\If{ \Big($ isCheaper(bestSolution) $
				or $ \exists SeasonPass(Trip) $ 
				and $ \nexists SeasonPass(UpperCategoryTrip) $\Big) }{
					$ categoryPromotion(Trip)$\;
			}
		}
	}
	$ Solution\_List \leftarrow join(Suitable\_List, Valid\_Alternatives\_List)$\;
	$ Solution\_List.add(carSolution, bikeSolution)$\;
	
	\bigskip
	\tcp{Car is used during the Day}
	\If{Calendar contains Trips with $Car$}{
		$ Solution\_List.rankUp(carSolution)$\;
	}
	
	\bigskip
	\tcp{Weather Accounting}
	\If{$WeatherForecaster.Time() \leq 15$ days and $WeatherForecaster.Prediction()$ is $bad\_Weather$}{
		$ Solution\_List.remove(bikeSolution)$\;
		\ForAll{Solution in $Solution\_List$}{
			$ outdoorTime \leftarrow 0$\;
			\ForEach{SubTrip in Solution that is 'outdoor'}{
				$ outdoorTime \leftarrow outdoorTime + SubTrip.Time$\;
			}
			\If{ $outdoorTime \geq 3$ minutes}{
				$ Solution\_List.rankDown(Solution)$\;
			}
		}
	}
	
	\bigskip
	return $Solution\_List$\;
\end{algorithm}
	\vfill

	
	
%ENSURE BREAKS %	
\subsection{Ensure Break Time}
	This Algorithm show how to ensure that, when a new Appointment is inserted, that it all the Specified Break Time are ensured (See RASD for more Details).
	In order to apply this algorithm, the Appointment is supposed to be inserted during the \textit{Time Span} of a Break
	
	Given the actual Calendar, including Appointment, Trips and Breaks of a particular Day (see figure \ref{calendarManagerDetail}), the insertion of a new Appointment is \textbf{supposed not to be overlapped with other Appointments}.
	
	With respect to the new Appointment, 
	\begin{itemize}
		\item[-] If a previous Appointment exists, the estimated time to reach the Location from the event is added before the $newAppointment.StartHour$.
		\item[-] If the next Appointment is right after the 'End Hour', the estimated time to reach the new Location is added after the $newAppointment.EndHour$.
		\item[-] In the Other case this time is added before the 'Start Hour' of the next Event.
	\end{itemize}
	
	Then, the new Appointment is added, to the Timeline of the Day and a new Break is calculated in order to be consistent:
	\begin{itemize}
		\item[-] if the new Appoiment $StartHour$  and $EndHour$ are far from the $Break\_Start$ and $Break\_End$ the Break is left alone and the Algorithm returns previously computed Break.
		\item[-] A time interval $\delta t$ is determined, where the left extremity is the $Time\_Span\_Init$ or the $Appointment.EndTime$, while the right extremity is the $newAppointment.StartTripTime$.
		If $\delta t$ is greater or equal than Break's $Minimum\_Duration$, is added to a $Possible\_Break\_List$.
		
		\item[-] whether a possible Break is found or not, the process is reiterated with a new $\delta t$ where where the left extremity is the $newAppointment.StartTripTime$ or the $nextAppointment.EndTime$, until the right extremity is the $lastAppointment.StartTripTime$ or the $Time\_Span\_End$
 	\end{itemize}
 	
 	At the end, the maximum interval is took from $Possible\_Break\_List$ and it's set as $Default\_Break$
 	If $Possible\_Break\_List$ is empty the User is Notified that the new Appointment can't add that appointment because it doesn't let him have a Break.
 	
 	\vfill
	%\begin{algorithm}[H]
	\caption{Ensure Break Time}
	\KwData{Daily Calendar, New Appointmend}
	\KwResult{Daily Calendar consistent with breaks}
	
	\bigskip
	\tcp{Adding Extra Travel Time to New Appointment}
	
	\bigskip
	\tcp{Previous Appointmen before New Appointment}
	\If{$Calendar$ contains an Appointment $Event$ such that $Event.EndTime - NewAppointment.StartTime \leq 1 hour$}{
		$ timeSpan \leftarrow \textbf{Scheduler}.AVGTime(Event.Location, NewAppointment.Location)$\;
		$ NewAppointment.StartTime \leftarrow NewAppointment.StartTime - timeSpan$\;
	}
	\tcp{Next Appointmen is right after New Appointment}
	\If{$Calendar$ contains an Appointment $Event$ after $NewAppointment$}{
		$ timeSpan \leftarrow \textbf{Scheduler}.AVGTime(NewAppointment.Location, Event.Location)$\;
		\If{$NewAppointment.EndTime - Event.StartTime \leq 1 hour$}{
			$ NewAppointment.EndTime \leftarrow NewAppointment.EndTime + timeSpan$\;
		}
		\Else{
			$ Event.StartTime \leftarrow Event.Start - timeSpan$\;	
		}
	}
	
	\bigskip
	\tcp{Check Break Consistency}
	\ForAll{Breaks}{
		\If{Appointment's $StartTime$ or $EndTime$ are near Break's $StartTime$ or $EndTime$}{
			$ okFlag \leftarrow False$\;
		}
	}
	\If{okFlag}{
		\If{Added $timeSpan$}{
				removeTimeSpans()\;
			}
			$ \textbf{Calendar}.addAppointment(newAppointment)$\;		
			return $Calendar$\;
	}
	
	\bigskip
	\tcp{Check for New Breaks}
	
	\ForAll{EndTime, StartTime in Break.TimeSpan}{
		$\delta t = EndTime -StartTime $\;
		\If{$\delta t \geq Break.miminumDuration$}{
			$Possible\_Break\_List.add(EndTime, StartTime, \delta t )$\;
		}
	}
	
	\bigskip
	\If{$Possible\_Break\_List.isNotEmpty()$}{
		$ return \arg\max(Possible\_Break\_List)$\;
	}
	\Else{
		$\textbf{NotificationManager}.notify$ (Appointment doesn't permit Breaks)\;
	}
\end{algorithm}
	%\vfill

% REDUCE CO2 %