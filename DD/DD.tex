\documentclass[12pt, a4paper]{article}

\usepackage[utf8]{inputenc}
\usepackage[english]{babel}

% images %
\usepackage{graphicx} % to embed images
\graphicspath{ {IMG/} } % images are all in the same folder
\usepackage{subcaption} % for creating composite images
\usepackage[export]{adjustbox}
\usepackage{float}
% -- %

\usepackage{hyperref} % to link the table of contents
\usepackage{placeins} %floating
\usepackage{pdflscape} % to allow single pages in landscape mode
\usepackage[top=1.2in, bottom=1.2in, left=1in, right=1in]{geometry}

% include hyperlinks %
\usepackage{hyperref}
\hypersetup{
    colorlinks=true,
    linkcolor=black,
    urlcolor=blue,
}
% -- %

% algorithms %
\usepackage[ruled, vlined, linesnumbered]{algorithm2e} % algorithms
\usepackage{eurosym}
\usepackage{amssymb}
% -- %

\title{Design Document}
\date{2017-10-26}
\author{
	Leonardo Bisica
	\and
	Alessandro Castellani
	\and
	Michele Cataldo
}

\begin{document}
	%%% titlepage %%%
	\begin{titlepage}
		\centering
		\includegraphics[width=5cm]{img/polimi_logo}
		\vfill
		{\bfseries\Large
			Travlendar+\\
			Design Document\\
			Version 1.0\\
			\vskip4cm
			Leonardo Bisica\\
			Alessandro Castellani\\
			Michele Cataldo\\
		}
		\vfill
		\vfill
	\end{titlepage}

%%% TABLE OF CONTENTS %%%
	\tableofcontents
	
	
	
%%% 1 - INTRODUCTION %%%
	\newpage
	\section{Introduction}
		\subsection {Introduction}

Purpose 
The purpose of this Design Document is to explain the architecture of the Travlendar+ mobile application as well as the logic underlying its development. Our approach will be analyzed in detail section by section, keeping, above all, coherence with the path our RASD layed down.

Scoper
Travlendar+ is a mobile application that encompasses many different functionalities. 
It is first of all an event scheduler that keeps track of a user’s appointments so that it can display the fastest routes available indexed by transportation means thanks to external APIs.
Travlendar+ heavily relies on Google Maps APIs for tracking distances and travel times, but also uses the necessary APIs to locate and rent vehicles of sharing services and to buy public transportation tickets. Thanks to its ability to gather external info about weather, strikes and traffic, as well as average travel time, Travlendar+ can warn its users before creating an appointment (and during travel itself) if any overlap happens or if an event location can’t be reached in the expected time.

1.3 Definitions, Acronyms, abbreviations
We assume our Glossary already cover all the terms we introduced and specified in the RASD. In addition to that, we can add some additional word to our vocabulary :

Abbreviations :
[RWn] : The n^(th) runtime view

1.4 Reference Documents
- Mandatory Project Assignment, available in the BeeP page of the course
- (qualcosa potenzialmente sulla struttura di un DD, ma non c’è niente IEEE-related)

1.5 Document Structure
The document is organized into 7 sections.
Section 1  (Introduction) : the section at hand. It provides the scope of our project and frames our DD
Section 2 (Architectural Design) :  this section details the architecture of Travlendar+. It contains component, deployment and runtime view
Section 3 (Algorithm Design) :   this section displays the most important algorithms used by the mobile application
Section 4 (User Interface Design) : this section  builds upon the user interface built in the RASD of our project, detailing minor expansions to the interface which was already presented
Section 5 (Requirements Traceability) : this section tracks the requirements detailed in the RASD into their corresponding design elements
Section 6 (Implementation, integration and test plan) : this section identifies the order in which we implement the various subcomponents of the system as well as the order we want to implement and test them in
Section  7 (Reference) : this section accounts for the references of our project
Section 8 (Used Tools) : this section accounts for the tools we adopted in order to write down and deliver the DD

		

%%% 2 - ARCHITECTURAL DESIGN  %%%
	\newpage
	\section{Architectural Design}
		\subsection{Overview}

	As we anticipated, in this section we’re going to give an array of views of our system, shaping it from many angles at once. We’ll detail both the high-level components and the interfaces interleaved among them, delving then into deployment and runtime view.
	The architectural style we chose to adopt is a three-layered one.
	The decision to implement an external DB was born out of the need to provide a realiable and safe synchronization tool for our users and to store other kinds of personal information submitted through the app (like preferences and the like).

	
\subsection{Component View}
	The Component Diagram shown below describes the logical components of the system we are to develop, from a very high-level description on to a more detailed one. This diagram does not take into account the deployment phase, hence it doesn’t describe the logical layer of the system in terms of the physical tiers where it is deployed.

\subsubsection{High Level Component View}

	\begin{figure}[H]
			\centering
			\includegraphics[width = \textwidth]{UML/componentDiagrams/highLevel}
			\caption{High level view.}
			\label{componentHighLevel}
		\end{figure}

	We will now describe in detail all the components.

	\paragraph{Mobile Application}
		This component represents the view of the User over its system. It’s split in two sub-components, Guest-view and User-view, which represents the two different ways an human interaction can be instaurated with the system.

		\begin{figure}[H]
			\centering
			\includegraphics[scale = 0.2]{UML/componentDiagrams/mobileApplication}
			\caption{Mobile Application detail.}
			\label{mobileApplicationDetail}
		\end{figure}

	\paragraph{Access Manager}
	%TODO 
	BISOGNA SCRIVERE QUALCOSA QUI
	
		\begin{figure}[H]
			\centering
			\includegraphics[scale = 0.2]{UML/componentDiagrams/accessManager}
			\caption{Access Manager detail.}
			\label{accessManagerDetail}
		\end{figure}
		
		
	\paragraph{Application Aggregator}
		This component works, unsuprisingly, as a collector of the different information \textit{Travlendar+} manages. It allows an easy management of every piece of information and allows us to avoid an high number of interfaces among the different components. 'Profile Manager' specifies the profile setting of the current user, while 'User Action Handler' allows us to register User's input.
	
		\begin{figure}[H]
			\centering
			\includegraphics[scale = 0.2]{UML/componentDiagrams/applicationAggregator}
			\caption{Application Aggregator detail.}
			\label{applicationAggregatorDetail}
		\end{figure}
 
 
	\paragraph{Calendar Manager}
		This component is divided into 'Appointments' and 'Breaks' sub-components, which track the appointments inserted by the User together with his breaks, and 'Trips', which is the list of trips arranged by the scheduler for every appointment. The sub-component 'Appointment Aggregator' serves the purpose of listing and presenting the content of the other two sub-components.
	
		\begin{figure}[H]
			\centering
			\includegraphics[scale = 0.2]{UML/componentDiagrams/calendarManager}
			\caption{Calendar Manager detail.}
			\label{calendarManagerDetail}
		\end{figure}
 

	\paragraph{Preference Manager}
		This component serves the purpose of keeping track of the preferences expressed by the User. 'Season Pass Handler' takes care of storing season passes; 'Excluded Vehicles List' cuts off from the scheduler results involving a selection of banned transportation means, 'Preferences List' covers the remaining and wider spectrum of User's choices. 'Preference Handler' is the sub-component that manages the other ones and that communicates outside Preference Manager.

		\begin{figure}[H]
			\centering
			\includegraphics[scale = 0.2]{UML/componentDiagrams/preferenceManager}
			\caption{Preference Manager detail.}
			\label{preferenceManagerDetail}
		\end{figure}


	\paragraph{Travel Logic Manager}
		This component is split into two sub-components: 'Scheduler' is the fundamental block that aims at scheduling and arranging User appointments and breaks via the the corresponding trips, 'DistanceManager' is the block whose purpose is to organize and present travel times to the scheduler in order to have them sorted out and well-managed.

		\begin{figure}[H]
			\centering
			\includegraphics[scale = 0.2]{UML/componentDiagrams/travelLogic}
			\caption{Travel Logic detail.}
			\label{travelLogicDetail}
		\end{figure}
	

	\paragraph{Payment Manager}
		This component deals with the recording of purchases and their associated credit cards.
		The sub-component 'Payment Handler' tracks purchase records and interacts with the required apps installed on the mobile device, while 'Purchase History' is an exploitable and rational organizations of the credit cards used by the user.

		\begin{figure}[H]
			\centering
			\includegraphics[scale = 0.2]{UML/componentDiagrams/paymentManager}
			\caption{Payment Manager detail.}
			\label{paymentManagerDetail}
		\end{figure}
		
		
	\paragraph{Travlendar Server} 
		This component represents the \textit{Travlendar Server} whose purpose is to store User's preferences, access data and personal information. It is modeled by its 'DBMS' component, which stores Timetables and Preferences and the 'API Request Dispatcher', which forwards requests to external agents.

		\begin{figure}[H]
			\centering
			\includegraphics[scale = 0.2]{UML/componentDiagrams/travlendarServer}
			\caption{\textit{Travlendar Server} detail.}
			\label{serverDetail}
		\end{figure}


	\paragraph{API Manager} 
		This component is critical in order to provide a functioning Travel Logic: it gathers the interactions with all external APIs.
		Here are listed the APIs the mobile application project starts with : Google Maps API, Google Transit API, Open Weather Map API, Car2Go and BikeMi API.
		The last couple is for reference only, as already pointed in the RASD. Naturally, the list of external services can be expanded.
		The 'Listener' component serves the purpose of forwarding requests : it receives the desired inputs and standardized the formatting in order for our application to be easily read.
	
		\begin{figure}[H]
			\centering
			\includegraphics[scale = 0.2]{UML/componentDiagrams/APIManager}
			\caption{API Manager detail.}
			\label{APIManagerDetail}
		\end{figure}
	
	
	\paragraph{Notification Manager}
		This component allows the notification system to warn users in the cases events partially or completely overlap according to the scheduler.

	\paragraph{Localization Manager}
		This component represents the localization functionalities of the mobile application.


\subsubsection{Detailed Level Component View}

	\begin{landscape}
		\begin{figure}
			\includegraphics[height= 0.85\textheight]{UML/componentDiagrams/detailedLevel}
			\centering
			\caption{Detailed level view.}
			\label{detailedHighLevel}
		\end{figure}
	\end{landscape}

	
\subsection{Deployment View}
	\begin{figure}[H]
		\centering
		\includegraphics[width = 0.6\textwidth]{UML/deploymentDiagram/deploymentDiagram}
		\label{deploymentDiagram}
		\caption{Deployment Diagram}
	\end{figure}
	
\subsection{Runtime view}
	This section shows how the components run and interact in the system, by using Sequence Diagrams.

\subsubsection{Login}

	\begin{figure}[H]
		\centering
		\includegraphics[width=\textwidth]{UML/runtimeView/login}
		\caption{Login Sequence Diagram}
		\label{loginRunTimeView}
	\end{figure}
	
\subsubsection{Event Solutions Refreshing}

	\begin{figure}[H]
		\centering
		\includegraphics[width=0.9\textwidth]{UML/runtimeView/refreshEvent}
		\caption{Login Sequence Diagram}
		\label{refreshRunTimeView}
	\end{figure}

\subsection{Component Interfaces}
	In this section we will show the component Interfaces. For each one are reported the main functionalities.
Nevertheless we need to keep in mind that in further implementations these functions can be splitted into less complex ones.

\begin{figure}[H]

	\begin{subfigure}[t]{0.3\linewidth}
		\centering
		\includegraphics[width=\linewidth]{UML/Interfaces/mobileApplicationUserInterface}
		\caption{Mobile Application User Interface}
	\end{subfigure}
	~
	\begin{subfigure}[t]{0.3\linewidth}
		\centering
		\includegraphics[width=\linewidth]{UML/Interfaces/accessInterface}
		\caption{Access Interface}
	\end{subfigure}
	~	
	\begin{subfigure}[t]{0.3\linewidth}
		\centering
		\includegraphics[width=\linewidth]{UML/Interfaces/dataSharingInterface}
		\caption{Daba Sharing Interface}
	\end{subfigure}
	\hfill
	\vskip0.75cm
	
	
	\begin{subfigure}[t]{0.3\linewidth}
		\centering
		\includegraphics[width=\linewidth]{UML/Interfaces/calendarInterface}
		\caption{Calendar Interface}
	\end{subfigure}
	~
	\begin{subfigure}[t]{0.3\linewidth}
		\centering
		\includegraphics[width=\linewidth]{UML/Interfaces/appointmentSchedulingInterface}
		\caption{Appointment Scheduling Interface}
	\end{subfigure}
	~
	\begin{subfigure}[t]{0.3\linewidth}
		\centering
		\includegraphics[width=\linewidth]{UML/Interfaces/preferenceModificationInterface}
	\caption{Preference Modification Interface}
	\end{subfigure}
	\hfill
	\vskip0.75cm
	
	
	\begin{subfigure}[t]{0.3\linewidth}
		\centering
		\includegraphics[width=\linewidth]{UML/Interfaces/preferenceInfluenceInterface}
		\caption{Preference Influence Interface}
	\end{subfigure}
	~
	\begin{subfigure}[t]{0.3\linewidth}
		\centering
		\includegraphics[width=\linewidth]{UML/Interfaces/notificationInterface}
		\caption{Notification Interface}
	\end{subfigure}
	~
	\begin{subfigure}[t]{0.3\linewidth}
		\centering
		\includegraphics[width=\linewidth]{UML/Interfaces/programmaticInterface}
		\caption{Programmatic Interface}
	\end{subfigure}
\end{figure}	


\begin{figure}[H]\ContinuedFloat
	\begin{subfigure}[t]{0.3\linewidth}
		\centering
		\includegraphics[width=\linewidth]{UML/Interfaces/APICommunicationInterface}
		\caption{API Communication Interface}
	\end{subfigure}
	~
	\begin{subfigure}[t]{0.3\linewidth}
		\centering
		\includegraphics[width=\linewidth]{UML/Interfaces/localizationInterface}
		\caption{Localization Interface}
	\end{subfigure}
	~
	\begin{subfigure}[t]{0.3\linewidth}
		\centering
		\includegraphics[width=\linewidth]{UML/Interfaces/paymentInterface}
		\caption{Payment Interface}
	\end{subfigure}
	
	\caption{List of Interfaces}
\end{figure}

\subsection{Architectural Styles}
	For our system architecture we adopt a 3-tier client/server architecture : we'll have a Mobile Application Clients (the presentation layer) and Travlendar Server (which accounts with its separate functions both for Logic layer and Data Layer).


\paragraph{Mobile Application Clients:} This layer is represented by the \textit{Travlendar+} Application, a thick client and an interactive and dinamic GUI.
The application will be written in Java for increased portability and will be able both to geo-localize its user and to communicate with the core of the travel Logic: this is the foundation of information synching and requests forwarding to APIs.

\paragraph{\textit{Travlendar+} Server:} The business Server runs the business logic. Its main duty is to forward requests of the users to the Data layer and external APIs and to send them back bound by the necessary users' constraints. These operations are executed under a microservice architectural pattern.

\paragraph{The Data Layer:} This layer comprehends both the DBMS -which stores and manages users' preferences and timetables- and the external service agents (objects which call external services). In the former case it is required the capability of providing data through SQL queries using ODBC protocol, while in the latter it is required a good implementation of the ad-hoc functions of the external APIs.


\subsubsection*{Patterns} 

Pattern are mostly required to smooth the construction of our system. In particular, we'll be using a MVC (Model - View - Control) Pattern for our mobile application, Client/Server and Single Instance component.

MVC is used to define our Java mobile application: the view will be the dynamic GUI, the Travel Logic the control and the model will be the data received from the APIs and the locally stored timetables and trips.

Single Instance component will be mostly useful when treating the components of the mobile application: components like Access Manager or Travel Logic will be the ones mainly touched by this pattern.

Client/Server pattern is the one we adopted to shape the interactions between our mobile application (the client) and the Travlendar+ Server (i.e, the server). Reasons for adoption have mainly been the simplicity and widespread use of the model.

		
%%% 3 -ALGORITHM DESIGN  %%%
	\newpage
	\section{Algorithm Design}
		% RANK SOLUTIONS %
\paragraph{solution Ranking}
This algorithm shows how System ranks all the feasible solution based provided by the external sources, like Public Transportation, via Car, Bike or Foot, depending on:
	\begin{itemize}
		\item[•] time needed for completing the trip.
		\item[•] number of subtrips (only for Public Tranportation).
		\item[•] if the Car have to be used in other trips on that day.
		\item[•] money needed for the Tranportation Mean, calculated as:
			\begin{itemize}
				\item[-] public ticket cost for Public Services.
				\item[-] average fuel cost for Car.
				\item[-] tariff \euro /time for Sharing services.
			\end{itemize}
		\item[•] User preferences.
		\item[•] Weather forecast for the Appointment day (if and only if the Appointment is no longer than 15 days).
	\end{itemize}
	
	Since solutions have been already calculated individually, is supposed that every element in the input list has been already divided in subtrips either by the External API Manager or by Scheduler (see figure \ref{travelLogicDetail} and figure \ref{APIManagerDetail}) in a previous Step.
	Is also assumed that 'Public Service Manager' (see RASD Class Diagram) provides a list of different tranportation means, and all the consistent possible combinations of them.
	
	First of all Every element in the input list is filtered by the 'Excluded Vehicles List' (see figure \ref{preferenceManagerDetail})
	For remaining elements, every inner SubTrip time is compared with the average reference time that the User have to spend by going with owned Car or bike, or, if is a short distance, by going with foot, in order to categorize solutions in \textit{suitables}, \textit{valid alternatives}, and \textit{unconvenient}.
	
	
	
	Then for every category solutions:
	\begin{itemize}
		\item[-] items are orderd by \textit{time needed} and \textit{number of SubTrip}
		\item[-] if User has a Season Pass of a public transportation company, relative solutions are put on the highest rank.
		\item[-] solution cost are calculated, as aforementioned.
	\end{itemize}		
		
	If a solutions from lower category are advantageous with respect to 'higher' solutions, those are put again in the upper category.
	A solution is advantageous if \textit{time needed}s difference between the two category, defined as 
	\\ \quad $\Delta =  timeNeeded(bestUpperSolution) - timeNeeded(lowerSolution)$, \\
	is less than 15 minutes, and either is a cheaper solution or User has a Season Pass that doesn't belong to any other company of Upper-category solutions.
		
	If the Appointment is scheduled in less than 15 days, and weather forecast is predicted al \textit{non consistent for outdoor tripping}, solutions that expect a $total\_Outdoor\_Time$ defined as the sum of the time spent by walking and biking, greater than 3 minutes are downgraded in the respective category.

	If in the other Appointments of that day a proprietary Car solution is already scheduled, 'Only by Using Car' solution will be encouraged, but if and only if:

	In the end all the category \textit{suitables} and \textit{valid alternatives} are merged into an unique list made so that User can choose one.
	The algorithm return also the $bestSolution$ got by 'Poping' of the first element of the list, and the solution obtained by 'Only Walking' and 'Only by Using Car'.
		
\bigskip
	\begin{algorithm}[H]
	\caption{Rank Solution}
		\KwData{Trip, List of Calculated Solutions, Preferences, Calendar}
		\KwResult{List of Ranked Solutions, Best Calculated Solution, Only Car Time, Only Walking Time}
		
		\bigskip
		\tcp{Calculate 'Only Car' and 'Only Walking' Solutions}
		$ carTime \leftarrow \textbf{Scheduler}.carTime$\;
		$ bikeTime \leftarrow \textbf{Scheduler}.walkingTime$\;
		
		\bigskip
		\tcp{Filtering Solutions based on $Preferences$}	
		\ForAll{Trip in Input\_List}{
			\If{Trip.AssignedTransportationMean() is in Excluded\_Vehicles\_List}{
				$ InputList.remove(Trip) $\;
			}
		}
		$ FilteredList \leftarrow InputList $\;

		\bigskip
		\tcp{Assign a category}	
		\ForAll{Trip in Filtered\_List}{
			$ \delta_{Advantage} \leftarrow 0$\;
			\ForEach{SubTrip in Trip}{
				\Switch{SubTrip}{
					\Case{Long\_Distance}{
						$ AVG\_Ref \leftarrow 
							\quad \arg\min
								\Big( \textbf{Scheduler}.AVG\_Time(Car), \textbf{Scheduler}.AVG\_Time(Bike) \Big)$\;
					}
					\Case{Short\_Distance}{
						$ AVG\_Ref = \textbf{Scheduler}.AVG\_Time(Foot)$\;
					}
				}
				$ comparison \leftarrow AVG\_Ref - SubTrip.Time$\;
				$ \delta_{Advantage} \leftarrow \delta_{Advantage} + comparison$\;
			}
			
			\bigskip
			\If{$ \delta_{Advantage} \gg 0 $ }{ 
				$ Suitable\_List.add(Trip)$\;
			}
			\ElseIf{$ \delta_{Advantage} \geq 0 $ }{
				$ Valid\_Alternatives\_List.add(Trip)$\;
			}
			\ElseIf{$ \delta_{Advantage} < 0$ }{
				$ Unconvenient\_List.add(Trip)$\;
			}
			\Else{
				$ Filtered\_List.remove(Trip)$\;
			}
		}
		\bigskip
	\end{algorithm}
	
	%TODO trova il modo di spezzarli%
	
	\begin{algorithm}
			
		\tcp{Category Ranking}
		\ForEach{Category\_List}{
			$ Category\_List.sortBy(TimeNeeded, SubTripsNumber, ascending)$\;
			\ForAll{Trip in $Category\_List$}{
				$ cost \leftarrow calculateCost(Trip)$\;
				\If{User has Season Pass for that Trip}{
					$ cost \leftarrow SeasonPass.Promotion$\;
				}
			}
			$ Category\_List.partialOrdering(cost) $ \;
		}
		
		\bigskip
		\tcp{Searching for Advantageous Lower Solutions}
		$ bestSolution \leftarrow Suitable\_List.takeFirst()$\;
		\ForEach{lowerTrip in Lower\_Categories\_List}{
			$\Delta \leftarrow  timeNeeded(bestSolution) - timeNeeded(LowerTrip)$\;
			\If{$\Delta \leq 15$ minutes}{
				\If{ \Big($ isCheaper(bestSolution) $
					or $ \exists SeasonPass(Trip) $ 
					and $ \nexists SeasonPass(UpperCategoryTrip) $\Big) }{
						$ categoryPromotion(Trip)$\;
				}
			}
		}
		\bigskip
		\tcp{Car is used during the Day}
		
		\bigskip
		\tcp{Weather Accounting}
	\end{algorithm}

%%% 4 - USER INTERFACE DESIGN %%%
	\newpage
	\section{User Interface Design}
		This section is a reference to section 3.1.1 "User Interfaces" already seen in the RASD.
It was decided not to introduce new interfaces nor mockups as the existing ones are quite exhaustive with regards to the goals. In particular, it was already shown how interfaces were going to look like : we refer to mockups dealing with  “login”, “create an appointment”, “weekly view” and all possible solutions to reach an event. 
Notice that these are the goals the essential application functions from user's perspective
		
		
%%% 5 - REQUIREMENTS TRACEABILITY %%%
	\section{Requirement Traceability}
		\input{subsections/section5/requirementsTraceability.tex}

%%% 6 - IMPLEMENTATION, INTEGRATION AND TEST PLAN %%%
	\newpage
	\section{Implementation, Integration and Test Plan}
		
		\subsection{Integration Strategy}
			Sequence of component/function integration

It's time for us to detail the integration order of our components, detailing what we just anticipated with our dependency diagarms.
We'll procede in a bottom-up fashiong : first of all we'll detail the integration plan of the sub-components closely followed by the one of the components.


Component : API Manager

\textit{Internal integration strategy} : Bottom - Up
\textit{Integration order :}
	\begin{itemize}
		\item OpenWeatherMap
		\item Google Maps API
		\item Google Transit API
		\item Car2Go Api
		\item Other API-Based System
		\item Listener
	\end{itemize}


Component : Travlendar Server

\textit{Internal integration strategy} : Critical - First
\textit{Integration order :}
	\begin{itemize}
		\itemize DBMS
		\itemize	 API Request Dispatcher
	\end{itemize}
	

Component : Access Manager

\textit{Internal integration strategy} : Critical - first
\textit{Integration order :}
	\begin{itemize}
		\item Authentication Manager
		\item SignUp Handler
		\item Saved Login Data
	\end{itemize}
		
Component : Application Aggregator

\textit{Internal integration strategy} : Bottom - Up
\textit{Integration order :}
	\begin{itemize}
		\item Profile Manager
		\item User Actions Handler
	\end{itemize}
	
Component : Calendar Manager

\textit{Internal integration strategy} : Bottom - Up
\textit{Integration order :}
	\begin{itemize}
		\item Trip List
		\item Break List
		\item Appointment Aggregator
	\end{itemize}
	
Component : Preference Manager

\textit{Internal integration strategy} : Bottom - Up
\textit{Integration order :}
	\begin{itemize}
		\item Season Pass Handler
		\item Excluded Vehicles List
		\item Preferences List
		\item Preference Handler
	\end{itemize}
	

Component : Travel Logic

\textit{Internal integration strategy} : Bottom - Up
\textit{Integration order :}
	\begin{itemize}
		\item Trip Handler
		\item Scheduler
	\end{itemize}
	
Component : Localization Manager doesn't need sub-components integration plan

Component : Notification Manager doesn't need sub-components integration plan

Component : Payment Manager

\textit{Internal integration strategy} : Bottom - Up
\textit{Integration order :}
	\begin{itemize}
		\item Credit Card List
		\item Payment Handler
	\end{itemize}

Component : Mobile Application

\textit{Internal integration strategy} : Critical - First
\textit{Integration order :}
	\begin{itemize}
		\item User View
		\item Guest View
	\end{itemize}
	

			
		\subsection{Individual Steps and Test Description}
			The components will be tested following the interfaces they share and that we previously described, testing core methods also in case of sub-components integration.
Our main goal is to spot any kind of faults and failures, focusing in particular on input domains, covering the restricted spectrum of system reactions.
Because of this, we'll detail the tests in easy-to-consult tables, each introduced by the components and subcomponents that are required.
Each table represents an interface method, and on two columns we'll detail each input and its expected result.
We'll follow a bottom-up approach that requires the creation of drivers from time to time.

\subsubsection{API Communication Service}
	
	\textbf{OpenWeatherMap API,Listener}\\
		\begin{tabular}{| p{0.3\textwidth} |p{0.6\textwidth}|}
			\hline
			\hline
			
			\multicolumn{2}{|c|} {getWeatherInfo() }\\
			\hline
			
			\textbf{Input} & \textbf{Effect}.\\
			\hline
			\hline
			
			NullArgument.		&		NullArgumentException is raised.\\
			\hline
			
			Invalid Location	.	&		OpenWeather services return the information cannot be retrieved and the Listener signals the location is wrong.\\
			\hline		
			
			Valid Location.		&		Listener retrieves the required data structure from OpenWeatherMap and formats it.\\
			\hline
			\hline
		\end{tabular}
	
	\vskip1cm
	
	\noindent	
	\textbf{GoogleMapsAPI, Listener}\\
		\begin{tabular}{| p{0.3\textwidth} |p{0.6\textwidth}|}
			\hline
			\hline
			
			\multicolumn{2}{|c|} {getTravelTime() }\\
			\hline
			
			\textbf{Input}.		&		\textbf{Effect}.\\
			\hline
			\hline
			
			NullArgument.	&		NullArgumentException is raised.\\
			\hline
			
			Invalid Start Location/Destination Location.		&		GMAPS APIs signals one (or both) the locations are wrong and the Listener echoes this back.\\
			\hline		
			
			Valid Start Location and Destination Location.	&	 Listener retrieves the required data structure from GMAPS API.\\
			\hline
			\hline
		\end{tabular}

	\vskip1cm

	\noindent
	\textbf{Car2Go API, Listener}\\
	\textbf{BikeMi, Listener}\\
	\textbf{ Other API-Based Systems, Listener}\\
		\begin{tabular}{| p{0.3\textwidth} |p{0.6\textwidth}|}
			\hline
			\hline
			
			\multicolumn{2}{|c|} {findNearestVehicle() }\\
			\hline
	
			\textbf{Input}.		&		\textbf{Effect}.\\
			\hline
			\hline

			NullArgument.	&		NullArgumentException is raised.\\
			\hline
	
			Invalid Location or Location out of the boundaries.		&		The external signals the location is out of its service boundaries.\\
			\hline		
		
			Valid Location.		&		Listener retrieves the required data structure from the external API and formats it in order for it to be navigable within the mobile application.\\
			\hline
			\hline
		\end{tabular}

	\vskip1cm

	\noindent
	\textbf{Car2Go API, Listener}\\
	\textbf{BikeMi, Listener}\\
	\textbf{Other API-Based Systems, Listener}\\
		\begin{tabular}{| p{0.3\textwidth} |p{0.6\textwidth}|}
			\hline
			\hline
			
			\multicolumn{2}{|c|} {getReservationConfirmation() }\\
			\hline
			
			\textbf{Input}.		&		\textbf{Effect}.\\
			\hline
			\hline
			
			NullArgument.		&		NullArgumentException is raised.\\
			\hline
			
			Invalid Reservation Code.		&		The external service can’t provide reservation data, and it sends only a message error, that Listener echoes back.\\
			\hline
			
			Valid Reservation Code.	&		Listener retrieves the required data structure from the external API and formats it in order for it to be navigable within the mobile application.\\
			\hline
			\hline
		\end{tabular}

	\vskip1cm

	\noindent
	\textbf{Google Transit API, Listener}\\
		\begin{tabular}{| p{0.3\textwidth} |p{0.6\textwidth}|}
			\hline
			\hline
			
			\multicolumn{2}{|c|} {getTrafficInfo() }\\
			\hline
			
			\textbf{Input}.		&		\textbf{Effect}.\\
			\hline
			\hline
	
			NullArgument.	&		NullArgumentException is raised.\\
			\hline

			Invalid Path.		&		GMAPS APIs can’t find the route and send back a message to document the impossibility of monitoring the traffic. Listener echoes this back.\\
			\hline

			Valid Path & Google Transit API provides the required information about traffic and Listener sends them back.\\
			\hline
			\hline
		\end{tabular}

	\vskip1cm

	\noindent
	\textbf{Listener, API Request Dispatcher}\\
	\begin{flushleft} %two tables are alligned to the left

		\begin{tabular}{| p{0.3\textwidth} |p{0.6\textwidth}|}
			\hline
			\hline
			
			\multicolumn{2}{|c|} {provideMap() }\\
			\hline

			\textbf{Input}.		&		\textbf{Effect}.\\
			\hline
			\hline

			NullArgument.		&		NullArgumentException is raised.\\
			\hline

			Start Location and End Location.		&		Path from Start Location until End Location is raised.\\
			\hline

			Location and Firm’s name.			&		Path from Location until nearest vehicle is raised.\\
			\hline

			Location and Day.		&		Weather information is shown.\\
			\hline
			\hline
		\end{tabular}	
		\\
		\vskip0.25cm		
		\begin{tabular}{| p{0.3\textwidth} |p{0.6\textwidth}|}
			\hline
			\hline
			
			\multicolumn{2}{|c|} {retrievePaymentInfo() }\\
			\hline
			
			\textbf{Input}.		&		\textbf{Effect}.\\
			\hline
			\hline
			
			NullArgument.		&		NullArgumentException is raised.\\
			\hline
			
			Valid parameters.		&		A request of payment is sent.\\
			\hline
			
			Invalid parametert.		&		InvalidArgumentException is raised.\\
			\hline
			\hline
		\end{tabular}
	\end{flushleft}				


\vfill		
\subsubsection{Travlendar Server}
	\textbf{API Request Dispatcher, Payment Handler}\\
		\begin{tabular}{| p{0.3\textwidth} |p{0.6\textwidth}|}
		\hline
		\hline
		
		\multicolumn{2}{|c|} {APIRequest() }\\
		\hline
		
		\textbf{Input}.		&		\textbf{Effect}.\\
		\hline
		\hline
		
		NullArgument.		&		NullArgumentException is raised.\\
		\hline
		
		Valid Argument.		&		A payment request is sent to the Server.\\
		\hline
		
		Invalid Argument.		&		InvalidArgumentException is raised.\\
		\hline
	\end{tabular}

	\vskip1cm

	\noindent
	\textbf{API Request Dispatcher, Scheduler}\\
		\begin{tabular}{| p{0.3\textwidth} |p{0.6\textwidth}|}
			\hline
			\hline

			\multicolumn{2}{|c|} {APIRequest() }\\
			\hline

			\textbf{Input}.		&		\textbf{Effect}.\\
			\hline
			\hline
			
			NullArgument.		&		NullArgumentException is raised.\\
			\hline
		
			Valid Combination of Location, Data and Vehicle’s Firm is allowed.		&		A request is sent to the respective dispatcher.\\
			\hline
			
			Invalid Argument.		&		InvalidArgumentException is raised.\\
			\hline
		\end{tabular}

	\vskip1cm

	\noindent
	\textbf{DBMS, Authentication Manager}\\
		\begin{tabular}{| p{0.3\textwidth} |p{0.6\textwidth}|}
			\hline
			\hline
	
			\multicolumn{2}{|c|} {requestUserCalendar() }\\
			\hline

			\textbf{Input}.		&		\textbf{Effect}.\\
			\hline
			\hline

			NullArgument.		&		NullArgumentException is raised.\\
			\hline
		
			Invalid Argument.		&		InvalidArgumentException is raised.\\
			\hline
	
			User Credentials.		&		An access request is sent to DBMS.\\
			\hline
			\hline
		\end{tabular}

	\vskip1cm

	\noindent
	\textbf{DBMS, Autentication Manager}\\
		\begin{tabular}{| p{0.3\textwidth} |p{0.6\textwidth}|}
			\hline
			\hline
			
			\multicolumn{2}{|c|} {requestUserPreferences() }\\
			\hline
			
			\textbf{Input}.		&		\textbf{Effect}.\\
			\hline
			\hline
		
			NullArgument.		&		NullArgumentException is raised.\\
			\hline
		
			Invalid Argument.		&		InvalidArgumentException is raised.\\
			\hline
			
			User Credentials.		&		An access request of user preferences is sent to DBMS.\\
			\hline
			\hline
		\end{tabular}

	\vskip1cm

	\noindent
	\textbf{DBMS, Autentication Manager}\\
		\begin{tabular}{| p{0.3\textwidth} |p{0.6\textwidth}|}
			\hline
			\hline
		
			\multicolumn{2}{|c|} {updateUserCalendar() }\\
			\hline
			
			\textbf{Input}.		&		\textbf{Effect}.\\
			\hline
			\hline
		
			NullArgument.		&		NullArgumentException is raised.\\
			\hline
	
			Invalid Argument.		&		InvalidArgumentException is raised.\\
			\hline
		
			User Credentials and New DataCalendar.		&		User Calendar is update.\\
			\hline
			\hline
		\end{tabular}
		
	\vskip1cm

	\noindent
	\textbf{DBMS, Autentication Manager}\\
		\begin{tabular}{| p{0.3\textwidth} |p{0.6\textwidth}|}
			\hline
			\hline
		
			\multicolumn{2}{|c|} {updateUserPreferences() }\\
			\hline
			
			\textbf{Input}.		&		\textbf{Effect}.\\
			\hline
			\hline
		
			NullArgument.		&		NullArgumentException is raised.\\
			\hline

			Invalid Argument.		&		InvalidArgumentException is raised.\\
			\hline

			User Credentials and New DataPreferences.		&		User Preferences is update.\\
			\hline
			\hline
		\end{tabular}

	\vskip1cm

	\noindent
	\textbf{DBMS, Autentication Manager}\\
		\begin{tabular}{| p{0.3\textwidth} |p{0.6\textwidth}|}
			\hline
			\hline
			
			\multicolumn{2}{|c|} {registerUser() }\\
			\hline
			
			\textbf{Input}.		&		\textbf{Effect}.\\
			\hline
			\hline

			NullArgument.		&		NullArgumentException is raised.\\
			\hline

			Invalid Argument.		&		InvalidArgumentException is raised.\\
			\hline

			Valid Combination of User Credential and User Information.		&		New User Profile is created.\\
			\hline
			\hline
		\end{tabular}

	\vskip1cm

	\noindent
	\textbf{DBMS, Autentication Manager}\\
		\begin{tabular}{| p{0.3\textwidth} |p{0.6\textwidth}|}
			\hline
			\hline
		
			\multicolumn{2}{|c|} {checkCredentialUser() }\\
			\hline
		
			\textbf{Input}.		&		\textbf{Effect}.\\
			\hline
			\hline

			NullArgument.		&		NullArgumentException is raised.\\
			\hline

			Invalid Argument.		 &			False.\\
			\hline	
	
			User Credentials insert		&		True.\\
			\hline		
			\hline
		\end{tabular}
		
		
\vfill
\subsubsection{Payment Manager}
	
	\textbf{Payment Handler, Purchase History}\\
		\begin{tabular}{| p{0.3\textwidth} |p{0.6\textwidth}|}
			\hline
			\hline
			
			\multicolumn{2}{|c|} {addPaymentMethod() }\\
			\hline
			
			\textbf{Input}.		&		\textbf{Effect}.\\
			\hline
			\hline
			
			NullArgument.		&		NullArgumentException is raised.\\
			\hline
			
			Credit Card Data and linked Transaction information.		&		Data is stored in Purchase History.\\
			\hline
			
			Invalid Data.		&		Nothing changes and the input is rejected.\\
			\hline
			\hline
		\end{tabular}

	\vskip1cm

	\noindent
	\textbf{Payment Handler, Purchase History}\\
		\begin{tabular}{| p{0.3\textwidth} |p{0.6\textwidth}|}
			\hline
			\hline
			
			\multicolumn{2}{|c|} {deletePaymentMethod() }\\
			\hline
			
			\textbf{Input}.		&		\textbf{Effect}.\\
			\hline
			\hline

			NullArgument.		&		NullArgumentException is raised.\\
			\hline
			
			Credit Card Data.		&		Credit Card Data and the linked transactions stored in Purchase History are deleted.\\
			\hline
			
			Invalid Input.		&		Everything is unchanged in Purchase History.\\
			\hline
			\hline
		\end{tabular}

	\vskip1cm

	\noindent
	\textbf{Payment Handler, Purchase History}\\
		\begin{tabular}{| p{0.3\textwidth} |p{0.6\textwidth}|}
			\hline
			\hline
			
			\multicolumn{2}{|c|} {showPaymentRecord() }\\
			\hline
			
			\textbf{Input}.		&		\textbf{Effect}.\\
			\hline
			\hline
	
			NullArgument.		&		The payment record is returned in the correct data structure.\\
			\hline
			
			Any Parameter.		&		InvalidArgumentException is raised.\\
			\hline
			
			Invalid Input.		&		InvalidArgumentException is thrown and everything is unchanged in Purchase History.\\
			\hline
			\hline
		\end{tabular}
	
	\vskip1cm

	\noindent
	\textbf{Payment Handler, Scheduler}\\
		\begin{tabular}{| p{0.3\textwidth} |p{0.6\textwidth}|}
			\hline
			\hline
			
			\multicolumn{2}{|c|} {addPayment() }\\
			\hline
			
			\textbf{Input}.		&		\textbf{Effect}.\\
			\hline
			
			NullArgument.		&		NullArgumentException is raised.\\
			\hline
			
			Valid Ticket.		&		A payment request is shown.\\
			\hline 
			
			Invalid Input.		&		invalidArgumentException is thrown and everything is unchanged in Purchase History.\\
			\hline
			\hline
		\end{tabular}


\vfill
\subsubsection{Preference Manager}

	\textbf{Preference Handler, Excluded Vehicle List}\\
		\begin{tabular}{| p{0.3\textwidth} |p{0.6\textwidth}|}
			\hline
			\hline
			
			\multicolumn{2}{|c|} {setExcludedVehicles() }\\
			\hline
			
			\textbf{Input}.		&		\textbf{Effect}.\\
			\hline
			\hline
			
			NullArgument.		&		NullArgumentException is thrown.\\
			\hline
			
			Argument not corresponding to a collection of vehicle objects.		&		InvalidArgumentException is thrown.\\
			\hline

			Valid collection of vehicle objects.		&		The list of excluded vehicles is updated to become as the input.\\
			\hline
			\hline
		\end{tabular}

	\vskip1cm

	\noindent
	\textbf{Preference Handler, SeasonPass Handler}\\
		\begin{tabular}{| p{0.3\textwidth} |p{0.6\textwidth}|}
			\hline
			\hline
			
			\multicolumn{2}{|c|} {setSeasonPass() }\\
			\hline
			
			\textbf{Input}.		&		\textbf{Effect}.\\
			\hline
			\hline
			
			NullArgument.		&		NullArgumentException is thrown.\\
			\hline
			
			Argument not corresponding to a collection of seasonPasses.		&		InvalidArgumentException is thrown.\\
			\hline
			
			Valid collection of seasonPasses.		&		The list of seasonPasses is updated to become the same as the input.\\
			\hline
			\hline
		\end{tabular}

	\vskip1cm

	\noindent
	\textbf{Preference Handler, Preferences List}\\
	\begin{flushleft}

		\begin{tabular}{| p{0.3\textwidth} |p{0.6\textwidth}|}
			\hline
			\hline
			
			\multicolumn{2}{|c|} {setVehicleTimeSpan() }\\
			\hline
			
			\textbf{Input}.		&		\textbf{Effect}.\\
			\hline
			\hline
			
			NullArgument.		&		NullArgumentException is thrown.\\
			\hline
			
			Argument is not a collection of pairs of vehicles and time spans.		&		InvalidArgumentException is thrown.\\
			\hline

			Argument is a valid collection of pairs of vehicles and time spans.		&		The list of pairs of vehicles and time spans is updated to become the same as the input.\\
			\hline
			\hline
		\end{tabular}		
		\\
		\vskip0.25cm
		\begin{tabular}{| p{0.3\textwidth} |p{0.6\textwidth}|}
			\hline
			\hline
			
			\multicolumn{2}{|c|} {setCarbonFootprints() }\\
			\hline
			
			\textbf{Input}.		&		\textbf{Effect}.\\
			\hline
			\hline
			
			NullArgument.		&		Carbon footprints are not set.\\
			\hline
			
			Invalid non-integer argument.		&		InvalidArgumentException is thrown.\\
			\hline
			
			Valid integer input.		&		Carbon footprints are set as the input commands.\\
			\hline
			\hline
		\end{tabular}
		\\
		\vskip0.25cm
		\begin{tabular}{| p{0.3\textwidth} |p{0.6\textwidth}|}
			\hline
			\hline
			
			\multicolumn{2}{|c|} {setMaxDistancePerVehicle() }\\
			\hline
			
			\textbf{Input}.		&		\textbf{Effect}.\\
			\hline
			\hline
			
			NullArgument.		&		NullArgumentException is thrown.\\
			\hline
			
			Argument is not a collection of pairs of vehicles and integers.		&		InvalidArgumentException is raised.\\
			\hline

			Argument is a valid collection of pairs of vehicles and integers.		&		For each specified vehicle it is set the maximum allowed distance that it can travel.\\
			\hline
			\hline
		\end{tabular}
	
	\end{flushleft}

	\vskip1cm

	\noindent
	\textbf{PreferenceHandler, Scheduler}\\
		\begin{tabular}{| p{0.3\textwidth} |p{0.6\textwidth}|}
			\hline
			\hline
			
			\multicolumn{2}{|c|} {applyConstraints() }\\
			\hline
			
			\textbf{Input}.		&		\textbf{Effect}.\\
			\hline
			\hline
			
			NullArgument.		&		NullArgumentException is thrown.\\
			\hline
			
			Invalid Argoument.		&		InvalidArgumentException is thrown.\\
			\hline
			
			Valide combination of User Credentials and Type of constraint.		&		The list of trips is filtered according to existing preferences.\\
			\hline
			\hline
		\end{tabular}
	
	
\vfill
\subsubsection{Travel Logic}

	\textbf{Scheduler, Trip Handler}\\
		\begin{tabular}{| p{0.3\textwidth} |p{0.6\textwidth}|}
			\hline
			\hline
			
			\multicolumn{2}{|c|} {scheduleEvent() }\\
			\hline
			
			\textbf{Input}.		&		\textbf{Effect}.\\
			\hline
			\hline
			
			NullArgument.		&		NullArgumentException is thrown.\\
			\hline
			
			Argument is not a valid event paired with a location and a date.		&		InvalidArgumentException is thrown.\\
			\hline
		
			Argument is a valid event paired with a location and a date.		&		The event is added and scheduled by the logic of the mobile application.\\
			\hline
			\hline
		\end{tabular}

	\vskip1cm

	\noindent
	\textbf{Scheduler, Appointment Aggregator}\\
	\begin{flushleft}
		\begin{tabular}{| p{0.3\textwidth} |p{0.6\textwidth}|}
			\hline
			\hline
			
			\multicolumn{2}{|c|} {checkIntegrity() }\\
			\hline
			
			\textbf{Input}.		&		\textbf{Effect}.\\
			\hline
			\hline
			
			NullArgument.		&		NullArgumentException is thrown.\\
			\hline
			
			Argument is not a valid event paired with a location and a date.		&		False.\\
			\hline
		
			Argument is a valid event paired with a location and a date.		&		True.\\
			\hline
			\hline
		\end{tabular}		
		\\
		\vskip0.25cm		
		\begin{tabular}{| p{0.3\textwidth} |p{0.6\textwidth}|}
			\hline
			\hline
			
			\multicolumn{2}{|c|} {getTrips() }\\
			\hline
			
			\textbf{Input}.		&		\textbf{Effect}.\\
			\hline
			\hline
			
			Invalid Argument.		&		InvalidArgumentException is thrown.\\
			\hline
		
			Nothing.		&		All information about the trip.\\
			\hline
			\hline
		\end{tabular}
		\\
		\vskip0.25cm	
		\begin{tabular}{| p{0.3\textwidth} |p{0.6\textwidth}|}
			\hline
			\hline
		
			\multicolumn{2}{|c|} {calculateAverageTime() }\\
			\hline
	
			\textbf{Input}.		&		\textbf{Effect}.\\
			\hline
			\hline

			NullArgument.		&		NullArgumentException is thrown.\\
			\hline
			
			Invalid Argument.		&		InvalidArgumentException is thrown.\\
			\hline
			
			Valid combination of Data, Start Location and End Location.		&		Average Time of trip from Start Location to End Location.\\
			\hline
			\hline
		\end{tabular}
		
	\end{flushleft}

	
\vfill
\subsubsection{Calendar Manager}

	\textbf{Appointment Aggregator, Trip List}\\
		\begin{tabular}{| p{0.3\textwidth} |p{0.6\textwidth}|}
			\hline
			\hline

			\multicolumn{2}{|c|} {addEvent() }\\
			\hline

			\textbf{Input}.		&		\textbf{Effect}.\\
			\hline
			\hline
	
			NullArgument.		&		NullArgumentException is thrown.\\
			\hline
			
			Argument is not a valid event paired with a location and a date.		&		InvalidArgumentException is thrown.\\
			\hline
		
			Argument is a valid event paired with a location and a date.		&		The event is added to the events list, yet they miss trips, that must be added through Scheduler and Travel Logic.\\
			\hline
			\hline
		\end{tabular}

	\vskip1cm

	\noindent
	\textbf{Appointment Aggregator, Trip List}\\
		\begin{tabular}{| p{0.3\textwidth} |p{0.6\textwidth}|}
			\hline
			\hline
			
			\multicolumn{2}{|c|} {modifyEvent() }\\
			\hline
			
			\textbf{Input}.		&		\textbf{Effect}.\\
			\hline
			\hline
			
			NullArgument.		&		NullArgumentException is thrown.\\
			\hline
			
			Argument is not a valid field or set of fields of a valid event.		&		InvalidArgumentException is thrown.\\
			\hline
		
			Argument is a valid field or set of fields of a valid event.		&		The event is updated as requested.\\
			\hline
			\hline
	\end{tabular}

	\vskip1cm

	\noindent
	\textbf{Appointment Aggregator, Break List}\\
		\begin{tabular}{| p{0.3\textwidth} |p{0.6\textwidth}|}
			\hline
			\hline
			
			\multicolumn{2}{|c|} {setBreaks() }\\
			\hline
			
			\textbf{Input}.		&		\textbf{Effect}.\\
			\hline
			\hline
			
			NullArgument.		&		NullArgumentException is thrown.\\
			\hline
			
			Argument is not a valid break object.		&		InvalidArgumentException is thrown.\\
			\hline

			Argument is a valid collection of break objects.		&		The Break lists encompasses only the breaks listed in the input.\\
			\hline
			\hline
		\end{tabular}

	\vskip1cm

	\noindent
	\textbf{Appointment Aggregator, User Actions Handler}\\
		\begin{tabular}{| p{0.3\textwidth} |p{0.6\textwidth}|}
			\hline
			\hline
			
			\multicolumn{2}{|c|} {showAppointments() }\\
			\hline
			
			\textbf{Input}.		&		\textbf{Effect}.\\
			\hline
			\hline
			
			Invalid Argument.		&		InvalidArgumentException is thrown.\\
			\hline
		
			Nothing.		&		List of appointments.\\
			\hline
			\hline
		\end{tabular}

	
\vfill
\subsubsection{Access Manager}

	\textbf{Authentication Manager, Saved Login Data}\\
	\begin{flushleft}

		\begin{tabular}{| p{0.3\textwidth} |p{0.6\textwidth}|}
			\hline
			\hline
			
			\multicolumn{2}{|c|} {checkLocalAccount() }\\
			\hline
			
			\textbf{Input}.		&		\textbf{Effect}.\\
			\hline
			\hline
			
			NullArgument and no login data stored.		&		The standard login procedures that interleave with the Travlendar Server begin.\\
			\hline

			NullArgument and no login data stored.		&		The standard login procedure is bypassed and the user automatically acesses the mobile application.\\
			\hline
			\hline
		\end{tabular}
		\\
		\vskip0.25cm
			\textbf{Authentication Manager, SignUp Handler, Profile Manager,DBMS}\\
		\begin{tabular}{| p{0.3\textwidth} |p{0.6\textwidth}|}
			\hline
			\hline
			
			\multicolumn{2}{|c|} {signUp() }\\
			\hline

			\textbf{Input}.		&		\textbf{Effect}.\\
			\hline
			
			NullArgument.		&		NullArgumentException is raised.\\
			\hline
			
			InvalidArgument Exception.		&		InvalidArgumentException.\\
			\hline
			
			Valid combination of User Credentials and User Information.		&		User is signed up.\\
			\hline
			\hline
		\end{tabular}
		\\
		\vskip0.25cm
					\textbf{Authentication Manager, Profile Manager, DBMS}\\
		\begin{tabular}{| p{0.3\textwidth} |p{0.6\textwidth}|}
			\hline
			\hline
			
			\multicolumn{2}{|c|} {login() }\\
			\hline
			
			\textbf{Input}.		&		\textbf{Effect}.\\
			\hline
			
			NullArgument.		&		NullArgumentException is raised.\\
			\hline
			
			InvalidArgument Exception.		&		InvalidArgumentException.\\
			\hline
		
			Valid combination of User Credentials.		&		User is logged in.\\
			\hline
			\hline
		\end{tabular}

	\end{flushleft}

%%% 7 - APPENDIX %%%
	\newpage	
	\section{Appendix}
		\listoffigures
		\listofalgorithms
		
		\subsection{Used tools}
		For this assignment, we used the following tools:
		
		\begin{description}
			\item [LaTeX] The group used LaTeX to structure the final document and to help with versioning.
			\item [Github] We leaned on Github for versioning and coordinating synchronized work.
			\item[StarUML] We used StarUML  to make Use Case, Class and Sequence Diagrams. \href{http://staruml.io/}{StarUML}.
			
		\end{description}
		
		\subsection{Hours of work}
			\begin{description}
				\item[Bisica, Leonardo] around 44 hours of work;
				\item[Castellani, Alessandro] around 46 hours of work;
				\item[Cataldo, Michele] around 42 hours of work.
			\end{description}
			
\end{document}
