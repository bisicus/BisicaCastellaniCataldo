[2.1 Overview]
As we anticipated, in this section we’re goint to give an array of views of our system, shaping it from many angles at once. We’ll detail both the high-level components and the interfaces interleaved among them, delving then into deployment and runtime view .
[Check] The architectural style we chose to adopt is a three-layered one.
The decision to implement an external DB was born because of the need of providing a realiable and safe synchronization tool for our users and to store other kinds of personal information submitted through the app (like preferences and the like).
Since the client implements travel logic and actually consists in the mobile application we’re aiming to develop, we are without any doubt in the frame of a fat client design.


2.2 High Level Components and their interaction
An high-level perspective on our system : it details all the components.


Mobile Application
This component represents the view of the User over its system. It’s split in two sub-components, Guest-view and User-view, which represents the two different ways an human interaction can be instaurated with the system.

Application Aggregator
This component works, unsuprisingly, as a collector of the different information Travlendar+ manages. It allows an easy management of every piece of information and allows us to avoid an high number of interfaces among the different components. Profile Manager specifies the profile setting of the current user, while User Action Handler allows us to register User's input.
 
Calendar Manager
This component is divided into Appointments and Breaks sub-components, which track the appointments inserted by the User together with his breaks, and Trips, which is the list of trips arranged by the scheduler for every appointment. The sub-component Appointment Aggregator serves the purpose of listing and presenting the content of the other two sub-components.

Notification Manager
This component allows the notification system to warn users in the cases events partially or completely overlap according to the scheduler.

Localization Manager
This component represents the localization functionalities of the mobile application.  

Preference Manager
This component serves the purpose of keeping track of the preferences expressed by the User. Season pass handler takes care of storing season passes; Excluded Vehicles List cuts off from the scheduler results involving a selection of banned transportation means, Preferences List covers the remaining and wider spectrum of User's choices.
Preference Handler is the sub-component that manages the other ones and that communicates outside Preference Manager.

Travlendar Server

This component represents the Travlendar Server whose  purpose is to store User's preferences, access data and personal information. It is modeled by its DBMS component, which stores Timetables and Preferences and the API Request Dispatcher, which forwards requests to external agents.

Travel Logic Manager

This component is split into two sub-components : Scheduler is the fundamental block that aims at scheduling and arranging User appointments and breaks via the the corresponding trips, DistanceManager is the block whose purpose is to organize and present travel times to the scheduler in order to have them sorted out and well-managed.


Payment Manager

This component deals with the recording of purchases and their associated credit cards.
The sub-component Payment Handler tracks purchase records and interacts with the required apps installed on the mobile device, while Credit Card List is an exploitable and rational organizations of the credit cards used by the user.

API Manager

This component is critical in order to provide a functioning Travel Logic : it gathers the interactions with all external APIs. Here are listed the APIs the mobile application project starts with : Google Maps API, Google Transport API, Open Weather Map API, Car2Go and BikeMi API. The last couple is for reference only, as already pointed in the RASD. Naturally, the list of external services can be expanded.
The Listener component serves the purpose of forwarding requests and receives the desired inputs.




