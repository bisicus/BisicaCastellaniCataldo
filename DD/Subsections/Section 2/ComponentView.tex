\subsection{Component View}

2.1 Overview
As we anticipated, in this section we’re goint to give an array of views of our system, shaping it from many angles at once. We’ll detail both the high-level components and the interfaces interleaved among them, delving then into deployment and runtime view .
[Check] The architectural style we chose to adopt is a three-layered one.
The decision to implement an external DB was born because of the need of providing a realiable and safe synchronization tool for our users and to store other kinds of personal information submitted through the app (like preferences and the like).
Since the client implements travel logic and actually consists in the mobile application we’re aiming to develop, we are without any doubt in the frame of a fat client design.


2.2 High Level Components and their interaction
An high-level perspective on our system : it details all the components.


Mobile Application
This component represents the view of the User over its system. It’s split in two sub-components, guest-view and User-view, which represents the two different ways an human interaction can be instaurated with the system.

Application Aggregator
This component works, unsuprisingly, as a collector of the different information Travlendar+ manages. It allows easy storage of every piece of information and allows us to avoid an high number of interfaces among the different components. Profile manager specifies the profile setting of the current user, while User choice handler allows us to put constraints on the results provided by the scheduler.
 
Calendar Manager
This component is divided into Appointments&Breaks, which tracks the appointments inserted by the User together with his breaks, and Trips, which is the list of trips arranged by the scheduler for every appointment.


 

