\documentclass[12pt, a4paper]{article}

\usepackage[utf8]{inputenc}

\begin{document}

\begin{description}

\item[2.1 Overview]
As we anticipated, in this section we’re goint to give an array of views of our system, shaping it from many angles at once. We’ll detail both the high-level components and the interfaces interleaved among them, delving then into deployment and runtime view .
[Check] The architectural style we chose to adopt is a three-layered one.
The decision to implement an external DB was born because of the need of providing a realiable and safe synchronization tool for our users and to store other kinds of personal information submitted through the app (like preferences and the like).
Since the client implements travel logic and actually consists in the mobile application we’re aiming to develop, we are without any doubt in the frame of a fat client design.


2.2 High Level Components and their interaction
An high-level perspective on our system : it details all the components.


Mobile Application
This component represents the view of the User over its system. It’s split in two sub-components, guest-view and User-view, which represents the two different ways an human interaction can be instaurated with the system.

Application Aggregator
This component works, unsuprisingly, as a collector of the different information Travlendar+ manages. It allows easy storage of every piece of information and allows us to avoid an high number of interfaces among the different components. Profile manager specifies the profile setting of the current user, while User choice handler allows us to put constraints on the results provided by the scheduler.
 
Calendar Manager
This component is divided into Appointments and Breaks sub-components, which track the appointments inserted by the User together with his breaks, and Trips, which is the list of trips arranged by the scheduler for every appointment.

Notification Manager
This component allow the notification system to warn users in the cases events partially or completely overlap according to the scheduler.

Preference Manager
This component serves the purpose of keeping track of the preferences expressed by the User. Season pass handler takes care of storing season passes; Excluded Vehicls List cuts off from the scheduler results involving a selection of banned transportation means, Preference list covers the remaining and wider spectrum of User's choices.
Preference Modifier is the sub-component that manages the other ones and that communicates outside Preference Manager.

Travlendar+ Server

This component represents the Travlendar+ Server whose  purpose is to store User's preferences, access data and personal information. It is modeled by its DBMS component

Travel Logic Manager

This component is split into two sub-components : Scheduler is the fundamental block that aims at scheduling and arranging User appointments and breaks via the the corresponding trips, DistanceManager is the block whose purpose is to organize and present travel times to the scheduler in order to have them sorted out and well-managed.

Travlendar Logic Dependencies

This component covers a range of factors that must be taken into account for a good translation of our Functional Requirements and goals. WeatherForecaster 
TicketAcquirer deals with the purchase of tickets for public transportation means : it enables inter-app communication among the corresponding mobile applications.
Sharing Service Reserver : this sub-component links the Travlendar+ mobile application with the application corresponding to the desired service. It allows inter-app communication among the corresponding mobile applications. 

Payment Manager

This component deals with the procedures of ticket purchase

API Manager

This component is critical in order to provide a functioning Travel Logic : it gathers the interactions with all external APIs. Here are listed the APIs the mobile application project starts with : Google Maps API, Google Transport API, Open Weather Map API, Car2Go and BikeMi API. The last couple is for reference only, as already pointed in the RASD. Naturally, the list of external services can be expanded.
The API Requestes Dispatcher handles all the incoming requests from Travel Logic, while Listener receives and shapes the received data.



\end{description}
 





\end{document}