\documentclass[12pt, a4paper]{article}

\usepackage[utf8]{inputenc}
\usepackage{graphicx} % to embed images
\usepackage{hyperref} % to link the table of contents
\usepackage{subcaption} %complex images
\usepackage{placeins} %floating
%\usepackage{pdflscape} %to allow single pages in landscape mode
\usepackage[top=1.25in, bottom=1.25in, left=1in, right=1in]{geometry}
\usepackage{verbatim} % include raw text file (for Alloy)
\usepackage[export]{adjustbox}

\title{Requirement Analysis and Specification Document}
\date{2017-10-26}
\author{
	Leonardo Bisica
	\and
	Alessandro Castellani
	\and
	Michele Cataldo
}

\begin{document}
	%%% titlepage %%%
	\begin{titlepage}
		\centering
		\includegraphics[width=5cm]{img/polimi_logo.png} % also works with logo.pdf
		\vfill
		{\bfseries\Large
			Travlendar+\\
			Requirement Analysis and Specification Document
			Version 0.1\\
			\vskip4cm
			Leonardo Bisica\\
			Alessandro Castellani\\
			Michele Cataldo\\
		}
		\vfill
		\vfill
	\end{titlepage}

	%%% table of contents %%%
	\tableofcontents
	\newpage

	%%% introduction %%%
	\section{Introduction}
		\subsection{Purpose}
		The document that you are reading is the \textit{Requirement Analysis and Specification Document} (RASD) for information system \textit{Travelandar+}.
		This start-up aims to help users in their everyday-life between appointment’s organization and transportation. No previous version about this application have been developed. 
We will start from describe Stakeholders’ aims (\textit{Goals} Section), from which we obtain, subsequently, functional and nonfunctional requirements (\textit{Requirements} Section) useful to describe the system.
		In addition we will need other 2 sections to have a complete overview of the model: \textit{Constraints} Section that describes constraints about the system and \textit{Domain Properties} Section that underlines limits of the software imposed by the world.
		After spoke about this first chapter we could analyze some scenarios and use case of the application.
			
		\subsection{Actual system}
			da fare
		
		\subsection{Scope}
		The aim of this project is to develop \textit{Travlendar+}, a calendar based mobile application.
Main functionality of the system is helping people in scheduling their meetings, by giving them useful information like traffic informations, public transportation and weather.
	Meetings could be scheduled through the entire region of Lombardy (Italy), and the main Italian cities connected via railway system, and could be several type of work meetings and personal meetings.
Furthermore the system can help the user by buying via mobile application public transportation tickets, or by reserving a car or a bike of a Sharing system, if it’s available in the current city.
In case of bad weather the system should find alternative moving solutions in order to replace walking path.
Naturally, the system must also allow the registration of new users; for the registration the system requests both personal and payment information. After registration, the user could immediately start scheduling his meetings.
			
		%TODO	
		\subsection{Goals}
		\begin{enumerate}
	\item
		[G1] System allows guest user to register with an username ad and a password; to complete the procedure user should confirm by 

	\item
		[G2] System Login

	\item 
		[G3] Registered User can create meetings 
	\item 
		[G4] Registered User can edit meetings
	\item 
		[G5] The application can automatically compute a personalized selection of travel times between appointments to choose from
	\item
		[G6] User can choose a preferred solution among the best ones


\end{enumerate}	

		%TODO	
		\subsection{Domain properties}
		\input{Subsections/domain_properties.tex}
		%TODO	
		\subsection{Glossary}
		%\begin{description}
				\item[User] 
				\begin{itemize}
					\item First name;
					\item Last name; 
					\item Email;
					\item Username;
					\item Password;
					\item Payment information; this in particular includes:
						\begin{itemize}
							\item Credit card owner;
							\item Credit card number;
							\item Credit card expiration date;
							\item CVV number.
						\end{itemize}
				\end{itemize}
				
				\item[Guest] We shall call 'guests' all people who are using the interface of the system without being registered or logged in. Guests can't access any functionality of \textit{Travlendar+} except for the registration process and the log in. 
				
				\item[System]
				\item[Mobile Application]
				\item[Meeting Appintment]
				\item[Personal Appoitment]
				
\end{description}
		%TODO	
		\subsection{Assumptions}
		\subsection{Assumptions, dependencies, constraints}

	We've already given a formal and methodical definition of our problem, yet there are still some ambiguities which still need to be addresed.


		\begin{enumerate}
		
			\item Users do not create events outside the “operative zone”

			\item The devices Travlendar+ is installed on possess a well-functioning GPS for geo-localization

			\item If a registered User is willing to use the vehicle of a sharing network, we assume him to have downloaded the corresponding sharing-network app

			\item There’s no kind of dependency among the users of our system

			\item Any information coming from sharing services regarding position, rental, and payment won’t be double-checked by our system

			\item Information coming from external payment sites won’t be double checked by our system

			\item Buying public transportation tickets rely on the aforementioned external payment procedures

			\item System assumes the season passes submitted by the user only aid for filtering results and are not checked nor have any legal value

			\item System assumes that any car rental request is made by a person who’s allowed to make one. Only sharing services check the validity of documents; in addition to that we always assume the driver is the user who requests the rental

			
		\end{enumerate}

			
			


		
\end{document}


		\subsection{Constrains}			
		%\input{Subsections/constrains.tex}

		\subsection{Identifying stakeholders}


	\newpage
	\section{Actors identifying}
	%\input{Subsections/actors}
	\newpage
	\section{Requirements}
	%We now adopt a goal-based approach to determine the requirements associated with each one of the goals we have elaborated in Chapter 1.

We'll start numbering and exploring the goals we submitted.

\begin{itemize}
	
	\item \textit{[G1]}System provides an authentication system.
		\begin{itemize}
			\item [R.1.1] System provides sign-up and an authentication mechanism.
			\item [R.1.2] System requires a unique username and a password for every user.
			\item [R.1.3] An unregistered user is locked out the application and can only see registration page.
			\item [R.1.4] User has to confirm by mail his registration.
			\item [R.1.5] Only a correct combination of username and password will grant access.
             		\item [R.2.4] Application will implement a password retrieval mechanism.
			\item [R.1.6] Each modification made to a user account must be saved into Travlendar+ Server to be made effective.
			\item [R.1.7] New user registration is successful only after data is stored on Travlendar+ Server and a confirmation is received by the system.
			
		\end{itemize}
             
	                 

	\item \textit{[G2]} The application integrates a time-slot based system for appointments.
		\begin{itemize}
			\item [R.2.1] The calendar integrates a calendar and a timetable
			\item [R.2.2] Calendar must give to the user granularity regarding both months and days.
			\item [R.2.3] Calendar and Timetable can be modified only by the user inserting events. No one else is allowed to either see or modify the information they contain.
			\item [R.2.4] Calendar and Timetable for each user are remotely copied on Travlendar+ Server every time a user creates/modifies/deletes an event.
		\end{itemize}
                  
	
	\item \textit{[G3]} Registered User can create appointments.
		\begin{itemize}
			\item [R.3.1] User has to be registered and logged in the system in order to create an
appointment.
			\item [R.3.2] Appointments can be divided into work appointments (or meetings) and personal appointments
			\item [R.3.3] Appointments require a location, a starting time and an end time
			\item [R.3.4] Appointments location must be within the boundaries of the operative zone
			\item [R.3.5] There cannot be appointments with the same name, location and time
			\item [R.3.6] System must check suitability of created new entries based on already existing appointments
			\item [R.3.7] Appointment start time can't precede the actual system time at the moment of inserting it                                 							
			\item [R.3.8] User can select favourite travel means and priority for each appointment
			\item [R.3.9] Each appointment must be associated to a level priority
			\item [R.3.10] The creation of an appointment must be remotely saved on Travendlar+ server in order to be successful and complete
		\end{itemize}


	\item \textit{[G4]} Registered Users can edit appointments.
		 \begin{itemize}
		 	\item  [R.4.1] A modified meeting must respect all the constraints imposed during the creation of a new meeting, as the requirements in \textit{[G3]}.
		 	\item [R.4.2] A meeting can be modified up until its end time.
		 	\item [R.4.3] If the meeting is modified, the system behaves as if such an event was inserted for the first time, calculating all possibile conflicts with pre-existing events.
		 	\item [R.4.4] No limit actually exists on the amount of times an event can be modified within the aforementioned constraints.
		 	\item [R.4.5] A modification must be correctly saved on the remote \textit{Travlendar+} server in order to be succesful and completed.
		 	\item [R.4.6] Deleting an appointments must belong to the set of modifications.
		 \end{itemize}

	\item \textit{[G5]} The application can automatically compute a personalized selection of travel times between appointments to choose from.
		\begin{itemize}
			\item [R.5.1] The application must refer to Travel Logic for the expected travel time.
			\item [R.5.2] The application must be able to suggest a combination of various means to reach the desired destination.
			\item [R.5.3] In case the trip expects more than one travel mean, the journey must be divided into sub-problems whose expected travel time has to be calculated. Same goes with public means stop and shared vehicles.
			\item [R.5.4] Starting location for travel can be inserted manually, retrieved by the previous event or calculated through geo-localization.
			\item [R.5.5] The application must rank the suggestions according to their priority, presence of prefered travel means and time required.
			\item [R.5.6] The registered user must be able to choose to filter out specific travel means.
			\item [R.5.7] Favourite travel means associated to an appointment must always show up.
			\item [R.5.8] In case two or more appointments overlap, an appointment with higher priority is considered automatically chosen and all the remaining ones are arranged according to their priority. Warnings must follow as expected.
			\item [R.5.9] The route can include intermediate destinations before the final, target one.
			\item [R.5.10] When a shared vehicle is suggested the parking zone nearest to the destination must be always inserted among the intermediate destinations.
			\item [R.5.11] The sytem must grant to know daily scheduled times for public transportation through its APIs.
			\item [R.5.12] When the starting time of a trip associated to an event is only one hour away the system must notify the user with an updated list of travel time so he can choose.
			\item [R.5.13] According to real world data, each travel must have associated to itself the carbon footprints.
			\item [R.5.14] Travels that do not satisfy all User's contraints must be excluded.
		\end{itemize}
                  
                  
	\item \textit{[G6]} User can choose a solution among the scheduled ones. 
		\begin{itemize}
			\item [R.6.1] Selecting a solution that is not a personal vehicle must show both intermediate and final destinations.
			\item [R.6.2] The application must arrange a navigable interface of feasible solutions.
			\item [R.6.3] Choosing a solution that includes a public transportation mean must show the user the possibility to buy a ticket.
			\item [R.6.4] Choosing a solution that includes a shared vehicle must show the user the possibility to locate and rent such a vehicle.
			\item [R.6.5] Choosing a solution must not be definitive.
			\item [R.6.6] System must recognize by itself through geolocalization that a user reached destination; also, User must always be able to stop the trip.
		\end{itemize}
                  
                  
	\item \textit{[G7]} The application warns the user if locations are unreachable in the allotted time.
		\begin{itemize}
			\item[R.7.1] The application must realize if the alloted time is sufficient from either the last event, current location or manually inserted location.
			\item[R.7.2] The application must use as a reference the time to cover distance between the starting place and the destination one, using the futured scheduled time for public transportation if necessary.
			\item [R.7.3] Warning must arrive also while on the road if the travel mean is no longer suitable, or the best solution: in that case the system is going to prompt a new eventual choice of travel means.
			\item [R.7.4] When user reaches destination warnings must stop automatically.
			\item [R.7.5] Warnings can be disabled on the road by the user.
		\end{itemize}


	\item \textit{[G8]} Allow users to put constraints on different travel means and limit carbon footprints.
		\begin{itemize}
			\item[R.8.1] User must be able to rule out vehicles from search result returned by the system scheduler.
			\item[R.8.2] When the option of limiting carbon footprints gets enabled the associated CO2 consumed by each travel must be taken into account in travels scheduling.
			\item[R.8.3] User must be able to put a constraint on the number of travel means adopted for a single travel.			\item[R.8.4] User must allow at least a single travel mean.
			\item[R.8.5] User cannot remove "walking" from travel mean preferences.
		\end{itemize}


	\item \textit{[G9]} The application features additional User’s breaks.
		\begin{itemize}
			\item [R.9.1] Each Break is characterized by a duration, the time of the day they start in and by the time frame within are allowed.
			\item[R.9.2] Breaks can be periodic.
			\item[R.9.3] System reserves a minimum quantity of time which is not shorter than the break duration.
			\item[R.9.4] Breaks must be completely encapsulated within the time frames the break is allowed in	.
		\end{itemize}


	\item \textit{[G10]} The application allows to buy tickets for public services.
		\begin{itemize}
			\item[R.10.1] Buying a ticket must reroute the user to the corresponding payment service.
			\item[R.10.2] User must be able to choose among the different purchase options offered by the public transportation service provider.
		\end{itemize}


	\item \textit{[G11]} The application allows the nearest shared vehicle to be found and reserved.
		\begin{itemize}
			\item [R.11.1] A shared vehicle must necessarily belong to a bike-sharing service or a car-sharing service.
			\item [R.11.2] All services linked to shared vehicles must be automatically disabled if the location of an appointment out of the boundaries of the influence zone.
			\item [R.11.3] All sharing services have their own API which is used by the system to locate and reserve the vehicles.
			\item [R.11.4] To find or reserve a vehicle it's required that the user logins into the external corresponding service.
			\item [R.11.5] The external service can communicate with our mobile application. In case of reservation \textit{Travlendar+} checks if the mobile app corresponding to the desired services is installed on the system. All the following steps take place within such an environment, until control is returned to \textit{Travlendar+}.
			\item [R.11.6] The location of all the vehicles must be shown in the same interface, merging data from different APIs.
			\item[R.11.7] Only shared vehicles that are free and available must be displayed and possibly reserved.
		\end{itemize}


	\item \textit{[G12]} The application allows the user to oversee his position in real-time as well as the route of his travel.
		\begin{itemize}
			\item [R.12.1] Application integrates a map system submitted by Gmaps API.
			\item [R.12.2] User must be able navigate through the map indipendently from its current location.
			\item [R.12.3] User must be able search for a specific location.
			\item [R.12.4] The mobile device must be able to track its current position through geo-localization.
			\item [R.12.5] Positions of out the operative zone can't be accepted by the system and won't be displayed.
		\end{itemize}


	\item \textit{[G13]} The User can submit additional preferences
		\begin{itemize}
			\item[R.13.1] User must be able to forbid travel means within time spans, also periodical ones.
			\item[R.13.2] User must be able to put a constraint on the maximum amount of space and time he can give to each travel mean.
			\item[R.13.3] User must be able to link one or more season passes with his account.
			\item[R.13.4] User must be able to link one or more credit cards to his account.
			\item[R.13.5] Each modification apported by the User to its additional preferences is only made effective when synced on \textit{Travlendar+} Server.
		\end{itemize}
		
\end{itemize}
            
\vfill


	\newpage
	\section{Scenario identifying}
	%\subsection{Scenario "2"}

David is 17 years old and has recently joined the soccer team of his city. His coach has fixed that training sessions will be held every Monday, Wednesday and Friday at 18:00. Then David opens Travelandar+ and puts his commitments until the end of the season. The application suggests that the fastest way to go to the field training is the subway, so David decides to buy a season pass so that he can safely go to training sessions without having to ask his parents.\\
After buying season pass he registered it on the application.

% Latex non conta gli "a capo" che fai con "enter", ma solamente con il doppio backslash \\
% quindi sentiti libero di impagninarlo come meglio credi

\subsection{Scenario "3"}

Elizabeth loves dedicating the right amount of time to her appointments, from work to family to her hobbies. Recently though she’s having a hard time conciling all of her commitments. 
That’s why her friend Alex recommends her to use the Travelandar + application : Elizabeth follows his advice and downloads the application on her smartphone. 
It’s only half an hour and Elisabetta is very satisfied, especially because she could set up an "Optimal Lunch" function that allows her to devote the right amount of time to her lunch, denying the opportunity to add appointments around the 30 minutes dedicated to lunch.

\subsection {Scenario "4"}
John had an hard day, he’s just put the finishing touches on his project : he had to work even during this weekend, locked at home. Just when he’s done with his assignment he receives an invitation to go out and see a movie with Jane and the rest of his friends. 
He’s in a rush and he hasn’t previously registered such an appointment in his calendar; to make things worse, he doesn’t own a car, and public transportation is rather slow in the weekend. Because of this he rules out both car and the public transportation as travel means, and when he inserts location and time of the unexpected appointment only car sharing pops up as a viable and fast option. 
John obviously accepts and rents the car through Travlendar+. 


\subsection{Scenario "8"}

Tom is a bank employee working in Bologna. He decided to return at home in Milan next Friday to celebrate his father's birthday together with his family. Because of this he opens Travlendar+ on his smartphone and creates a "Dad's party" event for Friday night. Tom’s job does not allow him to leave Bologna before 18.00. Fortunately Travlendar+ also allows him to find travel solutions by cross-region trains as well as by car. In fact, it all comes down to Tom's choice. He proceeds to buy train tickets : to him Friday isn’t coming fast enough!
		

	\newpage
	\section{UML models}
	%\subsubsection{Use case diagram}
	A global picture of the system interaction with actors is provided here by means of use case diagrams. Following, an analysis of the most interesting use case situations derived from scenarios is presented.

	\includegraphics[width=\textwidth]{img/uml/useCase.png}

%%% CREATE A NEW EVENT %%%	
	\paragraph{Use Case 1: Create a New Event}
	
		\begin{tabular}{| l | p{0.8\textwidth} | }
			\hline
			\hline
			Actor	&		User. \\
			\hline
			Input Condition		&		User is already logged in into \textit{Travlendar+}. \\
			\hline
			Event Flow		&		\begin{enumerate}
												\item User click on "\textit{create event}".
												\item User set day, time, place, and event type.
												\item System checks if the new event overlap with already existing events or lunch period.
												\item	 System calculate, ranks and shows multiple solution depending on user travelling preferences.
												\item User select one solution as preferend one.
											\end{enumerate} \\
			\hline
			Output Condition		&		\textit{Tralvendar+} shows calendar main page with the new event. \\
			\hline		
			Exception		&		\begin{itemize}
											\item[-] Created event overlaps with already existing events.
											\item[-] There are no feasible solution.
										\end{itemize} \\
			\hline
			\hline
		\end{tabular}


%%% MODIFY EVENT %%%

	\paragraph{Use Case 2: Modify Event}
	
		\begin{tabular}{| l | p{0.8\textwidth} | }
			\hline
			\hline
			Actor	&		User. \\
			\hline
			Input Condition		&		\begin{itemize}
													\item[-] User is already logged in into \textit{Travlendar+}.
													\item[-] Event Already Exists.
												\end{itemize} \\
			\hline
			Event Flow		&		\begin{enumerate}
												\item User click on "\textit{Event}".
												\item User starts modifying process.
												\item System checks if the new event overlap with already existing events or lunch period.
												\item	 System calculate, ranks and shows multiple solution depending on user travelling preferences.
												\item User select one solution as preferend one.
											\end{enumerate} \\
			\hline
			Output Condition		&		\textit{Tralvendar+} shows calendar main page, with the updated event. \\
			\hline		
			Exception		&		\begin{itemize}
											\item[-] Created event overlaps with already existing events.
											\item[-] There are no feasible solution.
										\end{itemize} \\
			\hline
			\hline
		\end{tabular}
		
		
%%% INSERT PAYMENT METHOD%%%

	\paragraph{Use Case 3: Insert Payment Method}
	
		\begin{tabular}{| l | p{0.8\textwidth} | }
			\hline
			\hline
			Actor	&		User. \\
			\hline
			Input Condition		&		\begin{itemize}
													\item[-] User is already logged in into \textit{Travlendar+}.
													\item[-] Credit Card isn't already inserted on the system.
												\end{itemize} \\
			\hline
			Event Flow		&		\begin{enumerate}
												\item User click on "\textit{Preferences}" and then on "\textit{Payment Methods}".
												\item User sets all the credit cards info.
												\item System checks and validate provided informations.
											\end{enumerate} \\
			\hline
			Output Condition		&		\textit{Tralvendar+} returns to \textit{Payment Methods} page showing added card as a valid payment method. \\
			\hline		
			Exception		&		Credit card given informations are invalid. \\
			\hline
			\hline
		\end{tabular}

%%% BUY TICKET %%%

	\paragraph{Use Case 4: Buy Public Transportation Ticket}
	
		\begin{tabular}{| l | p{0.8\textwidth} | }
			\hline
			\hline
			Actor	&		User. \\
			\hline
			Input Condition		&		\begin{itemize}
													\item[-] User is already logged in into \textit{Travlendar+}.
													\item[-] A \textit{payment method} is already available.
												\end{itemize} \\
			\hline
			Event Flow		&		\begin{enumerate}
												\item User click on "\textit{Buy Ticket}".
												\item System shows available public trasportation services.
												\item User selects a ticket.
												\item	 System starts perchause transaction.
											\end{enumerate} \\
			\hline
			Output Condition		&		Based on public transportation service, User receives a valid ticket. \\
			\hline		
			Exception		&		Transaction doesn't work. \\
			\hline
			\hline
		\end{tabular}


% \input{Subsections/uml/class_diagram.tex}

% \input{Subsections/uml/sequence_diagram.tex}

	\newpage
	\section{Alloy modeling}
	%\subsection{The Model}
	The following model concerns the most characterizing features of the system. We avoided to burden the model with trivial and non-significant details.
	
	\subsubsection*{Data Types}
		\includegraphics[width = \textwidth]{alloy/code/dataType}

	\subsubsection*{Signatures}
		\includegraphics[width = \textwidth]{alloy/code/signature1}
		\vfill
		\includegraphics[width = \textwidth]{alloy/code/signature2}
	
	\subsubsection*{Facts}
		\includegraphics[width = \textwidth]{alloy/code/fact1}
		\vfill
		\includegraphics[width = \textwidth]{alloy/code/fact2}
	
	\subsubsection*{Predicates}
		\includegraphics[width = \textwidth]{alloy/code/predicate}
		\bigskip
		
\subsection{Results}
	\begin{figure}[H]
		\centering
		\includegraphics[width = 0.7\linewidth]{alloy/generatedWorld/Results}
		\caption{Result of the model analysis.}		
		\label{fig:alloyResult}
	\end{figure}
	\bigskip
	
	
	
	

\begin{landscape}
	\subsection{Generated World}
	
		\subsubsection{World Generated}
			\includegraphics[height= 0.9\textheight , center]{alloy/generatedWorld/WorldGenerated}
	
		\subsubsection{Ticket Purchase}
			\includegraphics[height= 0.9\textheight, width= 35cm, center]{alloy/generatedWorld/ticketPurchase}
	
		\subsubsection{Renting}
			\includegraphics[height = 0.9\textheight, width= 35cm, center]{alloy/generatedWorld/reserving}
		
		\subsubsection{Insert Appointment}
			\includegraphics[height = 0.7\textheight, width= 25cm, center]{alloy/generatedWorld/insertAppointment}

\end{landscape}
	
	


	\newpage
	%%% appendix %%%
	\section{Appendix}
		\listoffigures
		\listoftables
		
		\subsection{Used tools}
		For this assignment, we used the following tools:
		
		\begin{description}
			\item [Alloy] We used the alloy tool to write the code and check the models for the specification.
			\item [LaTeX] The group used LaTeX to structure the final document and to help with versioning.
			\item [Github] We leaned on Github for versioning and coordinating synchronized work.
			\item [Toggl] We used toggl to keep track of work hours.
			
		\end{description}
		
		\subsection{Hours of work}
			\begin{description}
				\item[Bisica, Leonardo] around xx hours of work;
				\item[Castellani, Alessandro] around xx hours of work;
				\item[Cataldo, Michele] around xx hours of work.
			\end{description}
			
\end{document}