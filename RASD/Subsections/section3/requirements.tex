We now adopt a goal-based approach to determine the requirements associated with each one of the goals we have elaborated in Chapter 1.

We'll start numbering and exploring the goals we submitted.

\begin{itemize}
	
	\item \textit{[G1]} System allows guest user to register with an username and a password; to complete the procedure user should confirm by e-mail.
		\begin{itemize}
			\item [R.1.1] System lets registering user choose an username and password.
			\item [R.1.2] Every username corresponds to a single user.
			\item [R.1.3] Duplicate usernames aren’t allowed.
			\item [R.1.4] Registering user can't be already registered.
			\item [R.1.5] An unregistered user is locked out the application and can only see registration page.
			\item [R.1.6] User has to confirm by mail his registration.
			\item [R.1.8] New user registration is successful only after data is stored on Travlendar+ Server and a confirmation is received by the system.
			\item [R.1.10] Each modification made to a user account must be saved into Travlendar+ Server to be made effective.
		\end{itemize}
             
	
	\item \textit{[G2]} System Login.
		 \begin{itemize}
		 	\item [R.2.1] User must be already registered to perform correct login.
		 	\item [R.2.2] User must remember username and password to login.
             \item [R.2.3] Only a correct combination of username and password will grant access.
             \item [R.2.4] Application will implement a password retrieval mechanism.
		\end{itemize}
                  

	\item \textit{[G3]} The application integrates a calendar and a timetable.
		\begin{itemize}
			\item [R.3.1] The calendar and the timetable must have two different interfaces.
			\item [R.3.2] Calendar must give to the user granularity regarding both months and days.
			\item [R.3.3] Calendar and Timetable can be modified only by the user inserting events. No one else is allowed to either see or modify the information they contain.
			\item [R.3.4] Calendar and Timetable for each user are remotely copied on Travlendar+ Server every time a user creates/modifies/deletes an event.
		\end{itemize}
                  
	
	\item \textit{[G4]} Registered User can create appointments.
		\begin{itemize}
			\item [R.4.1] User has to be registered and logged in the system in order to create an
appointment.
			\item [R.4.2] Appointments can be divided into work appointments (or meetings) and personal appointments
			\item [R.4.3] Appointments require a location, a starting time and an end time
			\item [R.4.4] Appointments location must be within the boundaries of the operative zone
			\item [R.4.5] There cannot be appointments with the same name, location and time
			\item [R.4.6] System must check suitability of created new entries based on already existing appointments
			\item [R.4.7] Appointment start time can't precede the actual system time at the moment of inserting it                                 							
			\item [R.4.8] User can select favourite travel means and priority for each appointment
			\item [R.4.9] Each appointment must be associated to a level priority
			\item [R.4.10] The creation of an appointment must be remotely saved on Travendlar+ server in order to be successful and complete
		\end{itemize}


	\item \textit{[G5]} Registered Users can edit appointments.
		 \begin{itemize}
		 	\item  [R.5.1] A modified meeting must respect all the constraints imposed during the creation of a new meeting, as the requirements in \textit{[G4]}.
		 	\item [R.5.2] A meeting can be modified up until its end time.
		 	\item [R.5.3] If the meeting is modified, the system behaves as if such an event was inserted for the first time, calculating all possibile conflicts with pre-existing events.
		 	\item [R.5.4] No limit actually exists on the amount of times an event can be modified within the aforementioned constraints.
		 	\item [R.5.5] A modification must be correctly saved on the remote \textit{Travlendar+} server in order to be succesful and completed.
		 	\item [R.5.6] Deleting an appointments must belong to the set of modifications.
		 \end{itemize}


	\item \textit{[G6]} The application can automatically compute a personalized selection of travel times between appointments to choose from.
		\begin{itemize}
			\item [R.6.1] The application must refer to Travel Logic for the expected travel time.
			\item [R.6.2] The application must be able to suggest a combination of various means to reach the desired destination.
			\item [R.6.3] In case the trip expects more than one travel mean, the journey must be divided into sub-problems whose expected travel time has to be calculated. Same goes with public means stop and shared vehicles.
			\item [R.6.4] Starting location for travel can be inserted manually, retrieved by the previous event or calculated through geo-localization.
			\item [R.6.5] The application must rank the suggestions according to their priority, presence of prefered travel means and time required.
			\item [R.6.6] The registered user must be able to choose to filter out specific travel means.
			\item [R.6.7] Favourite travel means associated to an appointment must always show up.
			\item [R.6.8] In case two or more appointments overlap, an appointment with higher priority is considered automatically chosen and all the remaining ones are arranged according to their priority. Warnings must follow as expected.
			\item [R.6.9] The route can include intermediate destinations before the final, target one.
			\item [R.6.10] When a shared vehicle is suggested the parking zone nearest to the destination must be always inserted among the intermediate destinations.
			\item [R.6.11] The sytem must grant to know daily scheduled times for public transportation through its APIs.
			\item [R.6.12] When the starting time of a trip associated to an event is only one hour away the system must notify the user with an updated list of travel time so he can choose.
			\item [R.6.13] According to real world data, each travel must have associated to itself the carbon footprints.
			\item [R.6.14] Travels that do not satisfy all User's contraints must be excluded.
		\end{itemize}
                  
                  
	\item \textit{[G7]} User can choose a solution among the scheduled ones. 
		\begin{itemize}
			\item [R.7.1] Selecting a solution that is not a personal vehicle must show both intermediate and final destinations.
			\item [R.7.2] The application must arrange a navigable interface of feasible solutions.
			\item [R.7.3] Choosing a solution that includes a public transportation mean must show the user the possibility to buy a ticket.
			\item [R.7.3] Choosing a solution that includes a shared vehicle must show the user the possibility to locate and rent such a vehicle.
			\item [R.7.4] Choosing a solution must not be definitive.
			\item [R.7.5] System must recognize by itself through geolocalization that a user reached destination; also, User must always be able to stop the trip.
		\end{itemize}
                  
                  
	\item \textit{[G8]} The application warns the user if locations are unreachable in the allotted time.
		\begin{itemize}
			\item[R.8.1] The application must realize if the alloted time is sufficient from either the last event, current location or manually inserted location.
			\item[R.8.2] The application must use as a reference the time to cover distance between the starting place and the destination one, using the futured scheduled time for public transportation if necessary.
			\item [R.8.3] Warning must arrive also while on the road if the travel mean is no longer suitable, or the best solution: in that case the system is going to prompt a new eventual choice of travel means.
			\item [R.8.4] When user reaches destination warnings must stop automatically.
			\item [R.8.5] Warnings can be disabled on the road by the user.
		\end{itemize}


	\item \textit{[G9]} Allow users to put constraints on different travel means and limit carbon footprints.
		\begin{itemize}
			\item[R.9.1] User must be able to rule out vehicles from search result returned by the system scheduler.
			\item[R.9.2] When the option of limiting carbon footprints gets enabled the associated CO2 consumed by each travel must be taken into account in travels scheduling.
			\item[R.9.3] User must be able to put a constraint on the number of travel means adopted for a single travel.
			\item[R.9.4] User must allow at least a single travel mean.
			\item[R.9.5] User cannot remove "walking" from travel mean preferences.
		\end{itemize}


	\item \textit{[G10]} The application features additional User’s breaks.
		\begin{itemize}
			\item [R.10.1] Each Break is characterized by a duration, the time of the day they start in and by the time frame within are allowed.
			\item[R.10.2] Breaks can be periodic.
			\item[R.10.2] System reserves a minimum quantity of time which is not shorter than the break duration.
			\item[R.10.4] Breaks must be completely encapsulated within the time frames the break is allowed in	.
		\end{itemize}


	\item \textit{[G11]} The application allows to buy tickets for public services.
		\begin{itemize}
			\item[R.11.1] Buying a ticket must reroute the user to the corresponding payment service.
			\item[R.11.2] User must be able to choose among the different purchase options offered by the public transportation service provider.
		\end{itemize}


	\item \textit{[G12]} The application allows the nearest shared vehicle to be found and reserved.
		\begin{itemize}
			\item [R.12.1] A shared vehicle must necessarily belong to a bike-sharing service or a car-sharing service.
			\item [R.12.2] All services linked to shared vehicles must be automatically disabled if the location of an appointment out of the boundaries of the influence zone.
			\item [R.12.3] All sharing services have their own API which is used by the system to locate and reserve the vehicles.
			\item [R.12.4] To find or reserve a vehicle it's required that the user logins into the external corresponding service.
			\item [R.12.5] The external service can communicate with our mobile application. In case of reservation \textit{Travlendar+} checks if the mobile app corresponding to the desired services is installed on the system. All the following steps take place within such an environment, until control is returned to \textit{Travlendar+}.
			\item [R.12.6] The location of all the vehicles must be shown in the same interface, merging data from different APIs.
			\item[R.12.7] Only shared vehicles that are free and available must be displayed and possibly reserved.
		\end{itemize}


	\item \textit{[G13]} The application integrates a map system.
		\begin{itemize}
			\item [R.13.1] The map must be submitted by an external service.
			\item [R.13.2] User must be able navigate through the map indipendently from its current location.
			\item [R.13.3] User must be able search for a specific location.
			\item [R.13.4] The mobile device must be able to track its current position through geo-localization.
			\item [R.13.5] Positions of out the operative zone can't be accepted by the system and won't be displayed.
		\end{itemize}


	\item \textit{[G14]} The User can submit additional preferences
		\begin{itemize}
			\item[R.14.1] User must be able to forbid travel means within time spans, also periodical ones.
			\item[R.14.2] User must be able to put a constraint on the maximum amount of space and time he can give to each travel mean.
			\item[R.14.2] User must be able to link one or more season passes with his account.
			\item[R.14.3] User must be able to link one or more credit cards to his account.
			\item[R.14.4] Each modification apported by the User to its additional preferences is only made effective when synced on \textit{Travlendar+} Server.
		\end{itemize}
		
\end{itemize}
            
\vfill
