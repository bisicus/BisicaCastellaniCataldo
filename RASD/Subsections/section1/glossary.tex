\begin{description}
	\item[User]
		We will refer to all people who are registered to the system as ’Users’. 
		All users have personal profiles which contain the following information:
		\begin{itemize}
			\item First name;
			\item Last name; 
			\item Email;
			\item Username;
			\item Password;
			\item Payment information; this in particular includes:
				\begin{itemize}
					\item Credit card owner;
					\item Credit card number;
					\item Credit card expiration date;
					\item CVV number.
				\end{itemize}
		\end{itemize}
				
		\item[Guest] We name 'guests' all the people who are using the interface of the system without being registered or logged in. Guests can't access any functionality of \textit{Travlendar+} except for the registration process and the log in. 
		
		\item[Operative Zone] We name Operative Zone the area within we can place the location of an event. For the time being the Operative Zone coincide with all the cities and places within italian peninsula that can be reached by simply consulting the Google APIs. Naturally, such an area may be expanded in the future.
				
		\item[Influence Zone] We name Influence Zone the area within whose borders the mobile application can not only give travel time by car and on foot (the minimum standard given to us by Google APIs) but also where \textit{Travlendar+} can rely at least on a single car and bike sharing service. For starting, the Influence Zone will coincide with the city of Milan.
				
		\item[Registered User] A registered User is a former guest who inserted his/her own credentials in the system. After previous login, a Registered User can then create events, work on the timetable and ultimately is the end-user of \textit{Travlendar+}.
				
		\item[Calendar] Calendar reflects the intuitive English meaning of the word. We mean by Calendar a month by month view within our mobile application and its implementation that has to be easy to export and modify.
				
		\item[Timetable] Calendar reflects the intuitive English meaning of the word. We mean by Timetable a day by day view within our mobile application and its implementation that has to be easy to export and modify.
			      
		\item[Vehicle Sharing services/systems and a Shared Vehicle] By vehicle sharing we do not inted referring to a generic 'car-pooling' service. The usage of a shared vehicle may be either one of two types, car or bike sharing. A vehicle within the system operates only within the boundaries and parking zones imposed by its service; it can be picked up by any user registered to its corresponding system, used for the required amount of time (even though a maximum time is always fixed) and then parked in an allowed zone, ready to be picked by up again by another user. Each sharing system possesses an individual API.
				
		\item[Break Time/Break] By break we refer to a privileged time span : it is a time span in which events can't be scheduled. It must have a minimum duration specified by the user and can be encapsulated in a bigger time frame.
				
		\item[Mobile Application] By mobile application we refer to a program conceived for Android and iOS operative systems, based on touch interfaces and able to run on portable devices. The logic of the mobile application is the system. We often abbreviate 'mobile application' into the more colloquial 'app'.
				
		\item[Appointment/Event] An appointment is an event well delimited both in time and space, requiring the presence and the direct investment, in our case, of the user who creates it. Appointments fall in two categories : work appointments (that are often referred also as meetings) and personal appointments (the broader set encapsulating all other kinds of appointments, mainly regarding personal and family life). It is also possible to encounter the word 'Event' meaning the same exact thing.
				
		\item[Warning] A warning is a notification given by \textit{Travlendar+} mobile application to the operating System it is hosted by. It behaves as a standard system notification.
				
		\item[Travel Logic] By travel logic we refer to the logic that processes the distances and the transportation time within our operative and influcence zones. In the case at hand, in this first implementation, we're going to adopt as Travel Logic the Google Maps APIs.
				
		\item [\textit{Travlendar+} Server] \textit{Travlendar} server is the simplified terminology we adopt to define the remote storage system indexed by user credentials. In such a server we can store users' perfonal preferences and timetables  
				
		\item [Travel means] We name travel means all possible travel solutions considered by our system : car, byke, public transportation, walking. All kinds of transporation on air and on sea are excluded.
				
\end{description}
